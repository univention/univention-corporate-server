\newcommand{\ucsUCRV}[1]{Univention Configuration Registry variable \ucsCommand{\ucsBCindex{#1}}}
%\newcommand{\ucsCVE}[1]{\href{http://cve.mitre.org/cgi-bin/cvename.cgi?name=CVE-#1}{CVE-#1}}
\newcommand{\ucsCVE}[1]{\href{http://security-tracker.debian.org/tracker/CVE-#1}{CVE-#1}}

\section{General}

\begin{itemize}

\item The Debian Squeeze 6.0.7 point update has been integrated. It
  provides many bugfixes (\ucsBug{30438}). The update also fixes
  some security issues:

\begin{itemize}
\item dbus-glib (\ucsCVE{2013-0292}) (\ucsBug{30496})
\item nss (\ucsCVE{2013-0743}) (\ucsBug{29939})
\item elinks (\ucsCVE{2012-4545}) (\ucsBug{29186})
\item exim4 (\ucsCVE{2012-5671}) (\ucsBug{29182})
\item emacs23 (\ucsCVE{2012-3479}) (\ucsBug{29146})
\item gnupg2 (\ucsCVE{2012-6085}) (\ucsBug{29928})
\item bacula (\ucsCVE{2012-4430}) (\ucsBug{29170})
\item openssh (\ucsCVE{2010-5107}, \ucsCVE{2011-5000}) (\ucsBug{29137})
\item tiff (\ucsCVE{2012-4564}, \ucsCVE{2012-5581}) (\ucsBug{29187})
\item libproxy (\ucsCVE{2012-4505}, \ucsCVE{2012-5580}) (\ucsBug{29178})
\item postgresql-8.4 (\ucsCVE{2013-0255}) (\ucsBug{30372})
\item libssh (\ucsCVE{2012-4559}, \ucsCVE{2012-4561}, \ucsCVE{2012-4562}, \ucsCVE{2012-6063}, \ucsCVE{2013-0176}) (\ucsBug{29348})
\item dbus (\ucsCVE{2012-3524}) (\ucsBug{29176})
\item proftpd-dfsg (\ucsCVE{2012-6095}) (\ucsBug{30073})
\item ffmpeg (\ucsCVE{2012-0858}, \ucsCVE{2012-2777}, \ucsCVE{2012-2801}, \ucsCVE{2012-2783}, \ucsCVE{2012-2784}, \ucsCVE{2012-2788}, \ucsCVE{2012-2803}) (\ucsBug{29153})
\item perl (\ucsCVE{2012-5195}, \ucsCVE{2012-5526}, \ucsCVE{2012-6329}) (\ucsBug{29180})
\item wireshark (\ucsCVE{2013-1582}, \ucsCVE{2013-1586}, \ucsCVE{2013-1588}, \ucsCVE{2012-4296}, \ucsCVE{2012-4048}, \ucsCVE{2013-1590}) (\ucsBug{29171})
\item gnupg (\ucsCVE{2012-6085}) (\ucsBug{29896})
\item nagios3 (\ucsCVE{2011-1523}, \ucsCVE{2012-6096}) (\ucsBug{29126})
\item openssl (\ucsCVE{2013-0166}, \ucsCVE{2013-0169}) (\ucsBug{30570})
\item apache2 (\ucsCVE{2013-1048}, \ucsCVE{2012-4557}, \ucsCVE{2012-3499}, \ucsCVE{2012-4558}) (\ucsBug{29184})
\item ghostscript (\ucsCVE{2012-4405}) (\ucsBug{29174})
\end{itemize}

\item The errata updates issued for UCS 3.1-0 have been integrated (\ucsBug{30178}).

\item For UCS 3.1-1 an independent installation DVD is available so
that all new installations already include bugfixes from the point and
errata updates up to 3.1-1 (\ucsBug{30506}).

\end{itemize}


\section{Univention Installer}
\begin{itemize}
\item The description of some keys has been corrected (this only applies to the English version) (\ucsBug{30528}).
\end{itemize}

%% \subsection{Profile-based installation}
%% \begin{itemize}
%% \end{itemize}


\section{Upgrade provisions (preup and postup scripts)}

\begin{itemize}

\item The update scripts \ucsName{preup.sh} and \ucsName{postup.sh} have
been adjusted to UCS 3.1-1 (\ucsBug{30622}, \ucsBug{30757}).

\item The preup script now saves the \ucsUCR{} settings before
the actual update (\ucsBug{19369}).

\item The filesystem check error message in the preup script has been
fixed (\ucsBug{30100}).

\end{itemize}

\section{Basic system services}

\subsection{Boot loader}
\begin{itemize}
\item The bootsplash background will be redrawn when a message is printed. 
      This fix avoids black lines during the boot (\ucsBug{29768}).
\end{itemize}

\subsection{Linux kernel and firmware packages}
\begin{itemize}
\item The Linux kernel and associated tools and firmware packages have
	been updated to version 3.2.39 (\ucsBug{30505}). The update also fixes
	a number of security issues (%
	\ucsCVE{2012-0957},
	\ucsCVE{2012-4398},
	\ucsCVE{2012-4461},
	\ucsCVE{2012-4508},
	\ucsCVE{2012-4530},
	\ucsCVE{2012-4565},
	\ucsCVE{2013-0190},
	\ucsCVE{2013-0216},
	\ucsCVE{2013-0217},
	\ucsCVE{2013-0228},
	\ucsCVE{2013-0871})
	and an issue with cloud instances (\ucsBug{29124}, \ucsBug{30369}).
\item The following packages have been backported to UCS 3.1-1 to provide support with Linux 3.2 kernels (\ucsBug{28978}, \ucsBug{30643}):
 \begin{itemize}
  \item Virtualbox 4.1.18
  \item OpenAFS 1.6.1
  \item iscsitarget 1.4.20.2-10
  \item ndiswrapper 1.57-1
  \item tp-smapi 0.41
  \item kbuild 0.1.9998
  \item scsitools 0.12-2.1
 \end{itemize}

\item A dependency to \ucsName{univention-kernel-headers} has been
  added to the package \ucsName{dkms} (\ucsBug{30785}).
\end{itemize}

\subsection{Univention Configuration Registry}
\begin{itemize}
\item \ucsName{config-registry.replog} now also contains the replaced value (\ucsBug{29855}).
\end{itemize}

%% \subsubsection{Changes to templates and modules}
%% \begin{itemize}
%% \end{itemize}

\subsubsection{Internal changes}
\begin{itemize}
\item Fixed the setting of apt options in the PackageManager class and add a function to mark
  packages as automatically installed (\ucsBug{29805}). Also the
  handling of dpkg output through pipes was corrected
  (\ucsBug{30370}) and dependency problems when installing multiple
  packages at once are now correctly resolved (\ucsBug{30279}).
\end{itemize}


\subsection{Network interface configuration}
\begin{itemize}
\item The file \ucsFile{resolv.conf} is now recreated if \ucsCommand{dhclient} has overwritten
the file. On a domain controller the nameserver provided via DHCP is configured as a DNS
forwarder (\ucsBug{29999}).
\end{itemize}


\subsection{Univention Firewall}
\begin{itemize}
\item Firewall scripts in \ucsFile{/etc/security/packetfilter.d/} may
  not contain a \emph{.} in their name, however, they may end on
  \emph{.sh} (\ucsBug{29706}).
\item Packetfilter exceptions have been added for Bacula (\ucsBug{25392}).
\end{itemize}


\section{Domain services}
\begin{itemize}
\item A regression was fixed in \ucsCommand{univention-backup2master},
  which prevented the transfer of Samba 4 FSMO roles
  (\ucsBug{29508}). The steps performed are now logged to
  \ucsFile{/var/log/univention/backup2master.log} (\ucsBug{29085}). If
  more than one backup domain controller existed in the domain, the forward and
  reverse lookup zones are now correctly set (\ucsBug{29242}).
\end{itemize}

\subsection{OpenLDAP}

\begin{itemize}

\item The attributes \ucsName{aRecord}, \ucsName{associatedDomain},
\ucsName{employeeNumber}, \ucsName{macAddress}, \ucsName{name}, \ucsName{ou},
\ucsName{pTRRecord}, \ucsName{relativeDomainName},
\ucsName{univentionInventoryNumber},
\ucsName{univentionOperatingSystem},
\ucsName{univentionSyntaxDescription},
\ucsName{univentionUDMPropertyLongDescription} and
\ucsName{univentionUDMPropertyShortDescription} have been added to the
recommended ldap substring index (\ucsBug{29509}).

\item The handling of additional LDAP servers configured through the
\ucsUCRV{ldap/server/addition} was fixed in \ucsName{univention-python} 
(\ucsBug{30436}).

\item A LDAP schema registration has been added which is still valid even after
the package uninstallation (\ucsBug{30596}).

\end{itemize}

%% \subsection{LDAP ACL changes}
%% \begin{itemize}
%% \end{itemize}

%% \subsection{LDAP schema changes}
%% \begin{itemize}
%% \end{itemize}

\subsection{Listener/Notifier domain replication}
\begin{itemize}
\item Fixed a segmentation fault during retrieval of binary attribute
  values from the cache (\ucsBug{30165}).
\item If the IP address was changed from DHCP to a fixed IP address in
  appliance setup mode, the listener and the notifier now correctly
  use the new address (\ucsBug{30408}, \ucsBug{30412}).
\item The debug level of several messages in the
  \ucsName{univention-directory-replication} package related to cache
  vs LDAP checks has been lowered (\ucsBug{26562}, \ucsBug{30521}).
\item The replication handler is now always run (\ucsBug{29475}).
\end{itemize}

\subsection{Domain joins of UCS systems}
\begin{itemize}


\item Various join scripts have been updated to use the library for
  join scripts (\ucsBug{28993}, \ucsBug{29424}):
  \begin{itemize}
    \item univention-apache
    \item univention-directory-listener
    \item univention-directory-notifier
    \item univention-directory-policy
    \item univention-heimdal
  \end{itemize}

\item The join script library now tests for the directory
  \ucsFile{/var/univention-join/joined} instead of a compatibility
  symlink (\ucsBug{28991}).

\item \ucsName{univention-run-join-scripts} no longer prints its header when
 called with parameters \emph{-dcaccount} and \emph{-dcpwd}
 (\ucsBug{29432}). Also, the redirected file descriptor during join
 script execution is now correctly closed. Otherwise it is possible
  that a running Python parent process won't return (\ucsBug{30245}).

\item \ucsName{univention-join} now logs if it was started with the
option \emph{-disableVersionCheck} (\ucsBug{30492}). The new option
\emph{-verbose} enables verbose logging of the join process to
\ucsFile{/var/log/univention/join.log} (\ucsBug{30154}).

\item The execution order of the join scripts in
\ucsCommand{univention-join}, \ucsCommand{univention-run-join-scripts} and
\ucsCommand{univention-check-join-status} has been fixed (\ucsBug{30168}).

\item \ucsCommand{univention-run-join-scripts} has been extended to run
specific join scripts with parameter \ucsCommand{--run-scripts} and to force
the execution of already executed scripts with the parameter
\ucsCommand{--force} (\ucsBug{30112}).

\item Support for unjoin scripts has been added (\ucsBug{30596}).

\item An invalidation of the nscd hosts cache has been added to
\ucsCommand{univention-join} (\ucsBug{30886}).

\end{itemize}


\section{Univention Management Console}

\subsection{Univention Management Console web interface}
\begin{itemize}
\item The usability of UMC on mobile devices (especially tablets) has been
  improved (\ucsBug{30167}, \ucsBug{30749}, \ucsBug{29622})

\item The menu structure has been improved with new icons in the top right
  corner (\ucsBug{25753}).

\item A link for feedback about UCS has been added (\ucsBug{25753}).

\item The version of the Dojo toolkit has been updated to 1.8.3. Related
  JavaScript packages has been updated to their latest release as well
  (\ucsBug{30167}, \ucsBug{30625}, \ucsBug{29210}).

\item An expired password can now be changed directly in the UMC login dialogue
  (\ucsBug{29971}).

\item Changed the link to the documentation to \ucsURL{http://docs.univention.de} (\ucsBug{29779})

\item The items on the start page are now sorted differently (\ucsBug{26858}, \ucsBug{29468})

\item Properly detect values that are automatically changed when
  editing an LDAP object (\ucsBug{29635}).

\item Icons of older UMC modules are now correctly displayed again (\ucsBug{29861}).

\item If logged in as \emph{root}, a note is displayed recommending to
  log in as the \emph{Administrator} instead (\ucsBug{26507}).

\item If the license is exceeded, a note is now displayed in the UMC overview
  page (\ucsBug{30355}). In addition a warning is shown if join scripts have
  not been executed or if the system has not been joined so far
  (\ucsBug{29489}).

\item Problems with permanently timed out sessions, especially with Internet Explorer,
  have been fixed (\ucsBug{30318}).

\item Support for anonymous user statistic has been added and can be controlled
  via \ucsUCRV{umc/web/piwik} (\ucsBug{30563}, \ucsBug{30574}, \ucsBug{30577},
  \ucsBug{30741}).

\item The width parameter in the URL query string had been ignored and
  is now applied to the UMC layout again (\ucsBug{30664}).

\item The UMC progressbar widget now uses fewer requests to update
  the progress information (\ucsBug{30151}).

\item The handling of disabled items in grids has been improved (\ucsBug{30795}).

\item The handling of validating widgets has been improved (\ucsBug{30109}).
\end{itemize}

\subsection{Univention Management Console server}
\begin{itemize}
\item Fixed UMCP responses which do not have a status because they are
  not in JSON format (\ucsBug{29957}).
\item Raised the log level of UMC (server and modules) to 2 (= PROCESS) as some
  important notifications are suppressed otherwise -- notably App Center
  installations and all tracebacks (\ucsBug{30033}).
\item The server debug output was extended (\ucsBug{29989}).
\end{itemize}

\subsection{Univention Management Console / Univention Directory
  Manager modules}

\subsubsection{Basic settings / Univention System Setup}
\begin{itemize}
\item The display of the progress bar in case of IP address changes
  has been corrected (\ucsBug{29435}).
\item Member servers are now again selectable as the system role in
  the appliance mode (\ucsBug{29759}).
\item A problem in the processing of locales was fixed, which could
  lead to an unusable language selection menu (\ucsBug{29569},
  \ucsBug{29970}, \ucsBug{29770}).
\item Update the list of recommended web browsers (\ucsBug{29771}).
\item If not using DHCP, the netmask of a network interface is set to
  255.255.255.0 by default (\ucsBug{29722}).
\item Log the dpkg output when (un)installing software components (\ucsBug{29239}).
\item Fixed a typo in the German translation (\ucsBug{29944}) and the
  title of the confirmation dialogue (\ucsBug{29817}).
\item The textmode fallback mode has been removed (\ucsBug{29718}).
\item If software components are selected to be installed on a system
  where no components are installed yet, no empty list of removed
  components will be added to the confirmation dialog anymore
  (\ucsBug{30603}).
\item The certificate page is now displayed only on the master domain
  controller (\ucsBug{30780}).
\item Enhanced the usability of the network page. It supports displaying of bridges, bonding and vlan devices now (\ucsBug{28389}).
\end{itemize}

\subsubsection{Users module}
\begin{itemize}
\item The empty value for the mail home server is appended, not
  prepended: The first server will be chosen by default while still
  allowing to set an empty value (\ucsBug{29635}).
\item The descriptions for locked login methods have been fixed
  (\ucsBug{19662}).
\item The labels of certain attributes in the certificate tab have
  been corrected and reordered (\ucsBug{30461}).
\item Setting the expiry interval (pwhistory policy) to 0 will
  deactivate the POSIX password expiry (\ucsBug{29918}).
\end{itemize}


%% \subsubsection{Extended attributes}
%% \begin{itemize}
%% \end{itemize}


\subsubsection{License module}
\begin{itemize}
\item The license import and check is now case insensitive
  (\ucsBug{29883}).
\item A typo in the detailed description of licenses has been
  corrected (\ucsBug{29862}).
\item The import of a license with a non-breaking space has been fixed
  (\ucsBug{30098}).
\end{itemize}


%% \subsubsection{Mail module}
%% \begin{itemize}
%% \end{itemize}

\subsubsection{System services module}
\begin{itemize}
\item The Cyrus service can now be administrated with the services
  module. For that, the \ucsUCRV{mail/cyrus/initscript} has been
  added. To keep the service administratable, it has to contain the 
  name of the cyrus init script. At installation, it defaults to
  \emph{cyrus-imapd}. If the init script of your Cyrus version has
  another name, you will have to change the variable (\ucsBug{29806}).
\item Bacula can now be managed (\ucsBug{17346}).
\item Squid can now be managed (\ucsBug{30295}).
\item Dansguardian can now be managed (\ucsBug{30296}).
\item The NFS kernel server can now be managed (\ucsBug{27783}).
\item ClamAV and Freshclam can now be managed (\ucsBug{13814}).
\item The Univention Virtual Machine Manager Daemon can now be managed (\ucsBug{29531}).
\item Fetchmail can now be managed (\ucsBug{30782}, \ucsBug{30781}).
\end{itemize}

\subsubsection{Domain join module}
\begin{itemize}
\item The join module has been revised with some new features (\ucsBug{27792}, \ucsBug{30112}):
  \begin{itemize}
    \item A progressbar is now shown during the join process.
    \item A system can be rejoined.
    \item The execution of already executed join scripts can be forced now.
  \end{itemize}
\end{itemize}

\subsubsection{Univention App Center}
\begin{itemize}
\item The user \emph{root} can no longer access the App Center (\ucsBug{30292}).
\item Added hardware requirements for an applicaton: Currently only a check
  for sufficient memory is available (\ucsBug{29113}).
\item Conflicts between packages found before installing an application are
  prompted to the user (\ucsBug{29598}). The removal of packages when
  uninstalling an application must now be confirmed (\ucsBug{29273}).
\item Allow an Apache restart during the installation of an application
  (\ucsBug{29808}, \ucsBug{29809}, \ucsBug{29810}).
\item Correctly display links to README and LICENSE of applications
  (\ucsBug{29875}).
\item Connect to a remote App Center instance over HTTPS (\ucsBug{29446}).
%% \item If an application requires an extension of the LDAP schema, a reminder
%%   was shown just before installation that the domain has to be prepared. This
%%   reminder is removed when installing on an DC master (\ucsBug{30037}).
\item Propagate the actual traceback in case an internal error occurs during the
  test whether an app can be installed/upgraded (\ucsBug{30777}).
\item Changed the message above the progressbar during App Center operations
  (\ucsBug{29761}).
\item Log the \ucsCommand{dpkg} output (\ucsBug{29239}).
\item Check the signature of packages before installing them (\ucsBug{29797}).
\item Track an upgrade of an application as \emph{update}, not \emph{install} (\ucsBug{29638}).
\item Allow requesting a new license when unable to \emph{upgrade} (\ucsBug{30169}).
\item Fixed an error when searching for packages apt knows about but which do
  not have a candidate in any repository (\ucsBug{29899}).
%% \item Corrected the statement before installing an application: The order in
%%   which LDAP-schema extensions are installed is important. They have to be
%%   installed first on the DC master and then on the DC backups (\ucsBug{30037}).
\item The start of the App Center was speeded by caching files locally (\ucsBug{30204}).
\item A new module was added showing details of every installed application
  (\ucsBug{30249}, \ucsBug{30561}, \ucsBug{30758}).
\item The details of an application were always the one of the newest version
  on the server, even when an older version was installed. This has been fixed
  (\ucsBug{30027}).
\item \ucsCommand{univention-add-app} now removes previously added versions of an app
  (\ucsBug{30384}). Installed applications are now tracked (\ucsBug{30433}).
  The correct app is now used (\ucsBug{29734}).
\item Whether an application can be upgraded was not checked accurately enough
  (\ucsBug{30385},\ucsBug{30637}).
\item The automatically included \emph{-errata} repository is no longer needed
  and no longer included (\ucsBug{30406}).
\item When an application is installed that includes an extension of the LDAP
  schema the corresponding packages are installed automatically on the 
  master domain controller and backup domain controllers by connecting
  to them via HTTPS (\ucsBug{30503}, \ucsBug{30663}).
\item Improved the confirmation dialog before installing an application by listing also
  packages that will be updated via automatic errata updates and all package
  changes that will be done on remote hosts (\ucsBug{30172}, \ucsBug{30769}).
\item If two App Center modules in the same domain have a significant
  version difference and connect to each other via HTTPS, a warning is written to the logs
  (\ucsBug{30662}).
\item The \ucsUCRV{repository/app\_center/server} is now set in postinst (\ucsBug{30264}).
\item The license request dialogue got an emphasised headline because it was
  confused with an error dialog (\ucsBug{30199}).
\item The App Center now removes all default packages during the deinstallation
 even if the packages are part of the default master packages (\ucsBug{30787}).

\end{itemize}

\subsubsection{Online update module}
\begin{itemize}
\item An inner scrollbar in the dialogue listing the packages to be
  updated was removed. It was only displayed when too many packages
  were listed (\ucsBug{30171}).
\end{itemize}


\subsubsection{Computers module}
\begin{itemize}
\item Fixed a failure in assigning DHCP objects to computers when the
  MAC address has a certain format (\ucsBug{30140}).

\item The host name of an IP managend client object can now be
  modified after initial creation (\ucsBug{7016}).

\item Nagios support can now be enabled for Mac OS X, Linux and Ubuntu
  computers (\ucsBug{30615}).

\item The Samba password is now set as well when the password is set
  via UDM/UMC (\ucsBug{30183}).

\item The search filter for Mac OS X clients has been fixed
  (\ucsBug{15729}).
\end{itemize}

%% \subsubsection{Policies}
%% \begin{itemize}
%% \end{itemize}

\subsubsection{Printers module}
\begin{itemize}
\item The syntax for the \emph{sambaName} Attribut has been changed. There
  are only letters, digits, dots and spaces allowed (\ucsBug{2857}).
\end{itemize}

\subsubsection{DHCP module}
\begin{itemize}
\item Fixed displaying the selectbox for object types in the advanced
  search (\ucsBug{29748}).
\item The syntax for the server identifier attribute of the DHCP
  statements policy has been changed to \emph{hostname or ip}
  (\ucsBug{19665}).
\end{itemize}

\subsubsection{UCR module}
\begin{itemize}
\item The description of a \ucsUCR{} variable is now shown in the module (\ucsBug{23223}).
\end{itemize}

\subsubsection{Policies}
\begin{itemize}
\item Policies are now sorted by their names (\ucsBug{24643}).
\end{itemize}

\subsubsection{LDAP directory browser}
\begin{itemize}
\item If only one LDAP object is to be (re)moved, show it in the dialog to
  avoid confusion; when right-clicking on the LDAP tree, act on the container
  under the cursor (\ucsBug{26062}).
\item The \ucsMenuEntry{UDM object} options were renamed to
  \ucsMenuEntry{LDAP object}  (\ucsBug{30529}).
\item After removing or moving a folder in the navigation tree, the next
  available parent node is now selected (\ucsBug{29400}).
\end{itemize}


\subsubsection{Other modules}
\begin{itemize}
\item Fix setting the password in the VNC module (\ucsBug{29951}).
\item Fixed a typo in the process overview (\ucsBug{27818}). Processes
  are now sorted by default (\ucsBug{30715}).
\item A bug in the computer and nagios modules has been fixed that
  could lead to a Nagios config mismatch (\ucsBug{30614}).
\item The descriptions for valid/invalid users/groups for file shares
  have been fixed (\ucsBug{7849}, \ucsBug{30714}).
\item When an user successfully changed his password using the
  \ucsMenuEntry{Change password} module, a notification will be
  displayed (\ucsBug{29251}).
\item Corrected button name and fixed quoting in the system info module (\ucsBug{29503}).
\end{itemize}

\subsection{Univention Directory Manager command line interface and related tools}
\begin{itemize}
\item \ucsName{univention-license-check} will now additionally print
  the base DN (\ucsBug{29100}).

\item Bash completion for \ucsCommand{univention-directory-manager ...
list} has been extended (\ucsBug{15245}).
\end{itemize}

\subsection{Development of modules for Univention Management Console}
\begin{itemize}
\item A bug in the MultiUploader widget has been corrected (\ucsBug{29960}).
\item The size and default value of the TimeBox widget have been corrected (\ucsBug{30093}).
\item Added generic support for widgets folding out in a confirmation dialogue
  (\ucsBug{30791}).
\item Fixed several errors in module templates (\ucsBug{29993}, \ucsBug{30164}).
\end{itemize}


\section{Software deployment}
\begin{itemize}
\item A deprecated call to \ucsName{univention-errata-update} has been
  removed from the \ucsName{univention-maintenance} cronjob
  (\ucsBug{29367}).
\item The repository path used by
  \ucsName{univention-repository-create} is now correctly determined
  from CDROM/ISO information (\ucsBug{29626}).
\item The package \ucsName{univention-errata-level} has been rebuilt
  so that the  \ucsUCRV{version/erratalevel} will be set to 0 (\ucsBug{30802}).
\end{itemize}

\subsection{Repository handling}
\begin{itemize}
\item The \ucsUCRV{repository/online/maintained} has been deprecated
  and is now enabled by default. The
  \ucsUCRV{repository/online/component/*/parts} has been deprecated in
  favour of the new
  \ucsUCRV{repository/online/component/*/unmaintained}, which can be
  used to explicitly includes the unmaintained section of the package;
  the default is inherited from
  \ucsUCRV{repository/online/unmaintained}. The previous settings are automatically migrated to the new variables (\ucsBug{30261}).
\item The \ucsName{univention-updater} now always configures
  \ucsURL{http://updates.software-univention.de/} as the initial
  repository server, because some external DNS servers always return a
  valid address for \emph{univention-repository.\$domainname}. If a
  local repository is used, the \ucsUCRV{repository/online/server}
  must be changed manually, best through a Repository Policy
  (\ucsBug{30409}).

\item The packages
\begin{small}
\begin{itemize}
\item \ucsName{python-yaml}, \ucsName{open-vm-dkms}, \ucsName{open-vm-tools}, \ucsName{php5-pspell}, \ucsName{php5-intl}, \ucsName{aspell-en},
\item \ucsName{ttf-liberation}, \ucsName{graphviz}, \ucsName{libauthen-sasl-perl}, \ucsName{libcarp-clan-perl}, \ucsName{libbit-vector-perl}, \ucsName{libdate-pcalc-perl},
\item \ucsName{libnet-domain-tld-perl}, \ucsName{libemail-valid-perl}, \ucsName{libmime-perl}, \ucsName{libunicode-string-perl}, \ucsName{libdbd-pg-perl},
\item \ucsName{libgd-gd2-noxpm-perl}, \ucsName{libgd-text-perl}, \ucsName{libgd-graph-perl}, \ucsName{libpdf-api2-perl}, \ucsName{libxml-parser-perl},
\item \ucsName{libnet-ssleay-perl}, \ucsName{libio-socket-ssl-perl}, \ucsName{libnet-imap-simple-perl}, \ucsName{libnet-imap-simple-ssl-perl},
\item \ucsName{libtext-glob-perl}, \ucsName{libnumber-compare-perl}, \ucsName{libfile-find-rule-perl}, \ucsName{libdata-compare-perl},
\item \ucsName{libconvert-asn1-perl}, \ucsName{libnet-ldap-perl}, \ucsName{libtext-csv-xs-perl}, \ucsName{libcommon-sense-perl},
\item \ucsName{libjson-xs-perl}, \ucsName{libio-pty-perl}, \ucsName{libipc-run-perl},  \ucsName{libxml-twig-perl}, 
\item \ucsName{libparse-recdescent-perl}, \ucsName{libgraphviz-perl}, \ucsName{libfont-afm-perl}, \ucsName{libhtml-format-perl},
\item \ucsName{libio-socket-inet6-perl}, \ucsName{libnet-libidn-perl}, \ucsName{libtie-ixhash-perl} and \ucsName{libxml-xpathengine-perl}
\item \ucsName{libcdt4}, \ucsName{libcgraph5}, \ucsName{libgraph4}, \ucsName{libpathplan4}, \ucsName{libxdot4}, \ucsName{libgvc5}
\item \ucsName{libgvpr1}, \ucsName{libxml-xxpathengine-perl}
\end{itemize}
\end{small}
have been added to the maintained section of the repository (\ucsBug{29867}, \ucsBug{30068}, \ucsBug{30039}, \ucsBug{30173}).
\end{itemize}

\subsection{Software deployment command line tools}
\begin{itemize}
\item The updater now uses an individual user agent string for the
  interaction via HTTP (\ucsBug{25489}).
\item \ucsCommand{univention-upgrade -{}-check} no longer updates a
  local repository (\ucsBug{30315}).
\end{itemize}

\subsection{Software monitor (univention-pkgdb)}
\begin{itemize}
\item The system's errata level is now saved in the package database
  (\ucsBug{30739}).
\end{itemize}

\section{Univention Library}
\begin{itemize}
\item The exception handling when searching for DNs via the python
  interface has been improved (\ucsBug{29499}).
\end{itemize}


\section{System services}

%% \subsection{DHCP}
%% \begin{itemize}
%% \end{itemize}

\subsection{DNS}
\begin{itemize}

\item Change the owner for \ucsFile{/etc/bind/rndc.key} to \emph{root} if Samba 4
is used as the Bind backend. This allows the use of the control interface
(rndc) (\ucsBug{25358}). Enabled rndc control in \ucsFile{named.conf.samba4} and restart Bind if
\ucsUCRVSA{dns/backend} is set to \emph{samba4} and the rndc reload fails (\ucsBug{30321},
\ucsBug{30657}).

\item UCR variables for the Bind debug level have been added
(\ucsUCRVSA{dns/debug/level} and \ucsUCRVSA{dns/dlz/debug/level})
(\ucsBug{29562}).

\end{itemize}

%% \subsection{Cyrus}
%% \begin{itemize}
%% \end{itemize}

%% \subsection{Postfix}
%% \begin{itemize}
%% \end{itemize}

\subsection{Spam/virus detection and countermeasures}
\begin{itemize}
\item Renamed the \ucsUCRV{mail/antivir/clamav-daemon/autostart} to
  \ucsUCRVSA{clamav/daemon/autostart} and the
  \ucsUCRV{mail/antivir/clamav-freshclam/autostart} to
  \ucsUCRVSA {clamav/freshclam/autostart}, which are now used in the
  Clamav init scripts (\ucsBug{29980}).
\end{itemize}

\subsection{Printing services}
\begin{itemize}
\item A bug in the evaluation of printer quota policies has been fixed (\ucsBug{29715}).
\item The creation of print groups in the Cups listener module has been fixed (\ucsBug{29742}).
\item Output redirection in the Cups init script has been fixed (\ucsBug{17663}).
\item The description of the \ucsUCRV{cups/cups-pdf/anonymous} has
  been fixed (\ucsBug{17549}). A description for
  \ucsUCRVSA{cups/autostart} has been added (\ucsBug{29669}).
\item Changed the default value for the \ucsUCRV{cups/server} to
  \emph{localhost} (\ucsBug{29699}).
\item \ucsCommand{univention-lpadmin} now uses \emph{-h localhost} as server if no
  server parameter was provided (\ucsBug{28476}).
\item If the new \ucsUCRV{cups/include/local} is set to \emph{true}, the
Cups configuration includes \ucsURL{/etc/cups/cupsd.local.conf} for
custom configurations (\ucsBug{19552}).
\end{itemize}

\subsection{Kerberos}
\begin{itemize}
\item Log messages from any abort are now printed (\ucsBug{29342}).
\item Starting with UCS 3.0 the Heimdal Kerberos \ucsCommand{kadmin} tool and the
  server assumed a default password lifetime of one year for all
  Kerberos principals that have a \emph{sambaPwdLastSet} in LDAP, overriding
  \emph{krb5PasswordEnd}. Heimdal Kerberos is now patched in order to honour
  \emph{krb5PasswordEnd} (\ucsBug{30589}).
\item A realm section for the NETBIOS domain has been added to
  \ucsFile{/etc/krb5.conf} on a Samba 4 domain controller (\ucsBug{28819}).
\item The version number of the package
  \ucsName{univention-python-heimdal} has been increased to avoid
  conflicts with old packages incorrectly versioned (\ucsBug{29747}).
\end{itemize}

\subsection{Proxy services}
\begin{itemize}
\item The \ucsUCRV{squid/allowfrom} can now be set to \emph{all} (\ucsBug{27523}).
\item Added the new \ucsUCRV{squid/append\_domain} (\ucsBug{10390}).
\item Authentication based on group memberships now supports groups with
  spaces in their names (\ucsBug{11819}).
\item Use CIDR masks in \emph{always\_direct} rule definitions
  (\ucsBug{25268}).
\item The \ucsUCRV{squid/auth/groups} has been removed, group
  authentication is handled by the \ucsUCRV{squid/auth/allowed\_groups}
  (\ucsBug{30553}).
\item Renamed the \ucsUCRV{squid/ldapauth/groups} to
  \ucsUCRVSA{squid/auth/allowed\_groups}, which is by default empty
  (\ucsBug{30553}).
\item Corrected the \ucsUCR{} variable description for
  \ucsUCRVSA{dansguardian/groups} (\ucsBug{25311}).
\item Set all UCR variables in postinst before restarting Dansguardian (\ucsBug{29494}).
\item Removed templates and ucr variables for \emph{banneduser} and
  \emph{exceptionuser} (\ucsBug{25327}).
\item Removed groupbased ip address ban and exceptionlists and added global settings in
  \ucsUCRVSA{dansguardian/bannedipaddresses} and
  \ucsUCRVSA{dansguardian/exceptionipaddresses} (\ucsBug{25321}).
\item Unused group config files are now deleted (\ucsBug{30270}).
\item The Dansguardian default group has been renamed from
  \emph{www-access} to \emph{defaultgroup} (\ucsBug{30553})
\end{itemize}

\subsection{Apache}
\begin{itemize}
\item Added the \ucsUCRV{apache2/ssl/ca} to managed the
  \emph{SSLCACertificateFile} and
  \ucsUCRVSA{apache2/ssl/certificatechainfile} to manage the
  \emph{SSLCertificateChainFile} configuration options (\ucsBug{26171},\ucsBug{29374}).
\end{itemize}

\subsection{Nagios}
\begin{itemize}
\item Updated the plugins \ucsName{check\_univention\_squid} and
  \ucsName{check\_univention\_dansguardian} to check for currently
  supported auth methods (\ucsBug{29506}).
\end{itemize}


%% \subsection{SSL}
%% \begin{itemize}
%% \end{itemize}

%% \subsection{NFS}
%% \begin{itemize}
%% \end{itemize}

\subsection{PAM / Local group cache}
\begin{itemize}
\item The template for \ucsFile{/etc/nsswitch.conf} was fixed to
  handle the \ucsUCRV{nss/group/cachefile} correctly, which can be used to
  disable group caching (\ucsBug{29916}).

\item A bug was fixed in the \ucsName{extrausers} NSS
  module, which caused groups to disappear from the cached LDAP data
  (\ucsBug{29915}).

\item The option \ucsName{maxent} was added to \ucsName{pam\_access}
  to configure the maximum buffer size needed to check group
  membership in large environments. The limit can be changed using the
  \ucsUCRV{pamaccess/maxent} (\ucsBug{29393}).

\item pam\_ldap is no longer used for password changes, only
  pam\_krb5 and pam\_unix (\ucsBug{29438}).

\item A join script has been added to \ucsName{univention-pam}
  (\ucsBug{25368}).
\end{itemize}


\subsection{Other services}
\begin{itemize}
\item Server and client backups are now discerned correctly, errors
  are now handled based on backup type (\ucsBug{25386}).

\item In case of an error remote-backup now writes information to
  stderr. A new variable \ucsName{BWLIMITKBPS} controls the maximum
  bandwidth used by rsync, default unlimited; rsync now compresses
  files during transmission (\ucsBug{5318}).

\item As a workaround for Windows w32tm time service behaviour the
  local stratum of the NTP server on an master domain controller
  and on other UCS roles was lowered to 5 and 9 respectively
  (\ucsBug{30198}).

\item The \ucsUCRV{ntp/tinker/panic} has been added. It can be used to
  configure the maximum time difference to the NTP server in seconds
  which is still synchronised. Default: 0 seconds, which means that
  any time difference is synchronised no matter how big it is
  (\ucsBug{25752}).
\end{itemize}


\section{Virtualisation}

\begin{itemize}
\item \ucsName{open-vm-tools} version 8.8 has been backported
 to provide VMware support with Linux 3.2 kernels
  (\ucsBug{29991}).
\end{itemize}

\subsection{libvirt}
\begin{itemize}
\item A bug in the Runit script of \ucsName{univention-libvirt} has
  been fixed, which prevented \ucsName{libvirtd} from being
  automatically restarted (\ucsBug{29667}).
\item A memory corruption issue has been fixed when snapshots are
  created or deleted (\ucsBug{30052}).
\item A crash in message dispatching on error paths was fixed
  (\ucsBug{30213}).
\item Snapshots of VMs with writeable raw images are no longer
  allowed, because by default the raw image is opened read-write,
  which is not snapshotable. Because of that floppy images are now by
  default attached read-only, which is changeable in the image
  settings. Empty drives are allowed for snapshots, too. Removing
  empty drives no longer asks for deleting the volume. Snapshot
  operations are still accessible even when non-snapshotable images
  are attached (\ucsBug{30472}).
\item In 32 bit environments \ucsName{libvirtd} failed to start the
  default storage pool (\ucsBug{29380}).
\item On upgrades of
  \ucsName{univention-virtual-machine-manager-node-kvm} the default
  network bridge for virtual machines was shutdown, which disconnects
  the interfaces of all runnings VMs. This is no longer done on
  upgrades (\ucsBug{30590}).
\item Certificates still using the MD5 algorithm - which
  is cryptographically broken and considered insecure - were rejected
  by libvirt. They are now allowed again (\ucsBug{30702}).
\end{itemize}

\subsection{Univention Virtual Machine Manager}
\begin{itemize}
\item UVMM failed to start when its internal cache files were
  truncated. This has been fixed (\ucsBug{30174}).
\item The usage of certain terms has been standardised (\ucsBug{22582}, \ucsBug{23427}).
\item A bug in the join script of
  \ucsName{univention-virtual-machine-manager-schema} has been fixed
  (\ucsBug{30757}).
\item The description field of a virtual machine in \ucsName{UVMM} is now displayed
  in a machine tooltip on the overview page (\ucsBug{24682}).
\item iSCSI storage pools are now supported (\ucsBug{19804}).
\end{itemize}

\subsection{Xen}
\begin{itemize}
\item The Linux kernel module \ucsName{xen-gntdev} is now
  automatically loaded by \ucsFile{/etc/init.d/xencommons}
  (\ucsBug{29581}).
\end{itemize}

%% \subsection{QEMU/kvm}
%% \begin{itemize}
%% \end{itemize}


\section{Desktop packages}
\begin{itemize}
\item It is now possible to login in GDM as \emph{root}. A warning
  will be displayed as the root user should only be used for
  administrative purpose (\ucsBug{28372}).
\item Added support for writing an empty device section in
  \ucsFile{xorg.conf} with the \ucsUCRV{xorg/device/driver}
  (\ucsBug{30799}).
\end{itemize}

\section{Services for Windows}

\begin{itemize}
\item A bug in the Samba 3 to Samba 4 migration in combination with
  very long LDAP DNs has been fixed (\ucsBug{29335}).
\item According to the Samba Team there have been cases, where Samba 3
  to Samba 4 migration failed to set \emph{userPrincipalName}, this has been
  fixed (\ucsBug{30013}).
\end{itemize}

\subsection{Samba 3}
\begin{itemize}
\item The Netlogon path can now be adjusted using the
  \ucsUCRV{samba/share/netlogon/path} (\ucsBug{29801}).
\item Fixed the order of UCR variable initialization, making a manual
  Samba restart after initial installation of the
  \ucsName{univention-samba} package unnecessary (\ucsBug{30271}).
\item Fixed a quoting error while parsing credential options in the
  join script (\ucsBug{28814}).
\item More Winbind logfiles are now covered by Logrotate (\ucsBug{29953}).
\item Added a new \ucsUCRV{samba/max\_log\_size} (defaults to 0) (\ucsBug{29542}).
\item Added a description for the \ucsUCRV{samba/interfaces} (\ucsBug{28014}).
\item The Samba 3 idmap secret file is now also changed after server
  password rotation (\ucsBug{30170}). In addition, the new machine
  password is now also saved to the local samba internal password store.
  The Samba internal weekly password rotation is now replaced by the regular
  periodic \ucsName{server\_password\_change} which runs every 21 days by default
  (controlled by \ucsUCRV{server/password/interval}) (\ucsBug{30539}).
\item A duplicate UCRWARNING header was removed in the
  \ucsFile{smb.conf} template subfile (\ucsBug{28426}).
\end{itemize}

\subsection{Samba 4}
\begin{itemize}
\item Samba has been updated to Version 4.0.3. The security fix for
  \ucsCVE{2013-1863} is included (\ucsBug{29755}).
\item During reinstallations of previously joined Samba 4 domain controllers
	the \ucsFile{98univention-samba4-dns} join script could abort due
	to an \emph{objectSid} conflict (\ucsBug{29083}). For
    installations made with the
	new UCS 3.1-1 DVD this works directly. If a UCS 3.1-0 Samba 4 DC needs to be
	reinstalled the \ucsFile{sam.ldb} in the directory
    \ucsFile{/var/lib/samba/private} on that domain controller
	must not contain any data at the time of join. This condition can be fulfilled
	by moving the directory out of the way on that domain controller
    before initiating the domain join.
\item The join script now also removes leftover Samba 4 DNS
    alias records pointing to the FQDN of the joining DC (\ucsBug{29504}).
\item During the join of a Samba 4 DC with \ucsUCRVSA{samba4/role} set
  to \emph{RODC} the \ucsFile{98univention-samba4-dns} join script
  failed (\ucsBug{29537}).
\item Fixed the \ucsName{ntpsigndsocket} path in \emph{ntp.conf} and
  file permissions to \ucsFile{/var/lib/samba/ntp\_signd}
  (\ucsBug{29688},\ucsBug{29688})
\item The Samba 4 nbt\_server was started even though the
  \ucsUCRV{samba4/service/nmb} was set to nmbd (\ucsBug{29865}).
\item Fixed the misnamed \ucsUCRV{samba4/service/nbtd} (\ucsBug{24868}).
\item Fixed a missing default for the \ucsUCRV{samba/debug/level} in the
  \ucsFile{smb.conf} template (\ucsBug{29961}).
\item The Netlogon path can now be configured using the
  \ucsUCRV{samba/share/netlogon/path} (\ucsBug{29801}).
\item Fixed the DRS replication in IPv6-only environments
  (\ucsBug{29526}).
\item Fixed a regression in the performance of the Samba 4 DNS server
  backend (\ucsBug{29985}).
\item Fixed access to file shares with the options \ucsName{force
  group} or \ucsName{valid users = @group} (\ucsBug{29553}, \ucsBug{29983}).
\item Fixed truncation of large read SMB requests triggering a bug in
  recent MacOS X clients causing target file corruption
  (\ucsBug{30007}).
\item Fixed a quoting error while parsing credential options in the
  join script (\ucsBug{28814}).
\item More Winbind logfiles are now covered by Logrotate (\ucsBug{29953}).
\item Added a new \ucsUCRV{samba/max\_log\_size} (defaults to 0) (\ucsBug{29542}).
\item Add a description for the \ucsUCRV{samba/interfaces}
  (\ucsBug{28014}).
\item Duplicate UCRWARNING header removed in \ucsURL{smb.conf}
  template subfile (\ucsBug{28426}).
\item Fixes for \ucsFile{check\_essential\_samba4\_dns\_records} were made
  (\ucsBug{28891}).
\item New script \ucsFile{purge\_s4\_computer.py} to manually remove a
  computer account from Samba 4 (\ucsBug{29460}).
\item Improve resilience of \ucsFile{s4search-decode} against broken
  supplementalCredentials (\ucsBug{28931}). Also print Active
  Directory timestamps in a human-readable format (\ucsBug{29992}).
\item Fix unreliable \ucsCommand{nmbd} restart (\ucsBug{30135}).
\item Unused GPOs are now moved to \ucsFile{/var/lib/samba/sysvol\_backup}
every night with the new tool
\ucsFile{/usr/share/univention-samba4/scripts/sysvol-cleanup.py}
(\ucsBug{27468}).
\item At least one Samba 4 domain controller should be configured as domain master browser.
With UCS 3.1-1 it can be configured via the \ucsUCRV{samba/domain/master}. On a 
master domain controller this option will be enabled during the update (\ucsBug{30132}).
\item A segfault in the Bind DLZ module has been fixed (\ucsBug{30716}).
\item Iteration over DCs and sites has been fixed in the
  \ucsCommand{check\_essential\_samba4\_dns\_records.sh} script
  (\ucsBug{30784}).
\end{itemize}

\subsection{Univention S4 Connector}
\begin{itemize}
\item The connector got stuck for a few minutes if the process used
  a network interface, which was shut down. The running connector will
  now be restarted if a network device is shutdown (\ucsBug{30119}).

\item The daemon function was called too late which could result in a
crash of the connector. This issue has been fixed (\ucsBug{30149}).

\item The group membership synchronisation could lead to an inconsistent
state. This issue has been fixed (\ucsBug{30317}).

\item A really fast import could overwrite the changes just made to an object
due to concurrent processes. This issue has been fixed (\ucsBug{30651}).

\item The connector now synchronizes the following computer objects:
 Mac OS X, Ubuntu, Linux and UCC (\ucsBug{29872}). Windows computer
 objects are changed to Mac OS X computer objects if the operating
 system LDAP attribute is set to \ucsName{Mac OS X} (\ucsBug{29998}).

\item The S4 connector network ifdown script has been moved from
\ucsURL{/etc/network/ifdown.d} to \ucsURL{/etc/network/if-post-down.d}
(\ucsBug{30414}).

\end{itemize}

\subsection{Univention Active Directory Connector}
\begin{itemize}

\item The group membership synchronisation could reach an inconsistent
state. This issue has been fixed (\ucsBug{29874}).

\item A really fast import could overwrite the changes just made to an object
due to concurrent processes. This issue has been fixed (\ucsBug{30652}).

\item If the \ucsFile{/var/lib/univention-connector/ad} runtime
  directory was stored on a ext2/ext3/ext4 partition, tracebacks could occur (\ucsBug{18125}).

\item The connector will only try to become a daemon once at start up
  (\ucsBug{30150}).

\item A warning that the LM hash could not be synced will only be
  logged if the \ucsUCRV{password/samba/lmhash} is set to \emph{true}
  (\ucsBug{29294}).
\end{itemize}


\section{ucs-test framework}
\begin{itemize}
\item Added new test \ucsFile{00\_base/25check-permissions-etc-secret}
  which looks for *secret files in \ucsFile{/etc} that are readable by
  group other (\ucsBug{29477}).

\item \ucsFile{26check\_logfiles\_general} now ignores
  \ucsFile{/var/log/dpkg.log} (\ucsBug{30718}).

\item \ucsFile{90change\_user\_pwd\_via\_udm} no longer checks if the
  master domain controller is reachable via SSH (\ucsBug{30733}).

\item The uniquemember test case has been adapted to UCS 3.1 (\ucsBug{12081}).

\item The https logging test case has been adapted to UCS 3.1 (\ucsBug{30734}).

\item Several \ucsName{ucs-test} cases concerning faillog have been corrected (\ucsBug{19016}).

\end{itemize}


\section{Other changes}
\begin{itemize}
\item Fixed an erroneous handling of the Python Notifier with command
  output. In certain cases this could lead to the loss of some output
  data \ucsBug{29549}).

\item A package dependency to \ucsName{univention-config} has been
  added to the package \ucsName{univention-ssh} (\ucsBug{28079}).

\item Unnecessary log messages in the listener modules
  \ucsFile{gencertificate.py} and \ucsFile{replication.py} have been
  removed (\ucsBug{9819}, \ucsBug{15567}).

\item A dependency to \ucsName{ldap-utils} has been added in
\ucsName{univention-config} (\ucsBug{28851}).

\item The usage message of the \ucsName{univention-runit} init script,
  which erroneously contained "univention-dhcp" instead of it's own
  name, has been corrected (\ucsBug{29383}).

\item A dependency to \ucsName{ntp} has been added to
  \ucsName{univention-server-member} (\ucsBug{30041}).

\item \ucsName{parted} now supports handling of the boot flag of the
protective MBR parition (\ucsBug{29449}).
\end{itemize}

%% \subsection{ucslint}
%% \begin{itemize}
%% \end{itemize}
