\newcommand{\ucsUCRV}[1]{Univention Configuration Registry variable \ucsCommand{\ucsBCindex{#1}}}

\section{General}
\begin{itemize}

\item The Debian Squeeze 6.0.6 point update has been integrated. It
  provides many bugfixes (\ucsBug{28490}). The update also fixes
  some security issues:

\begin{itemize}
\item libotr (CVE-2012-3461) (\ucsBug{28177})
\item devscripts (CVE-2012-2240,  CVE-2012-2241, CVE-2012-2242, CVE-2012-3500) (\ucsBug{28392})
\item nss (CVE-2012-0441) (\ucsBug{27488})
\item mono (CVE-2012-3382) (\ucsBug{27867})
\item libxslt (CVE-2011-1202 CVE-2011-3970 CVE-2012-2825) (\ucsBug{25686})
\item vte (CVE-2012-2738) (\ucsBug{27696})
\item apache2 (CVE-2012-2687) (\ucsBug{27730})
\item network-manager (CVE-2012-2736) (\ucsBug{25609})
\item postgresql-8.4 (CVE-2012-3488, CVE-2012-3489) (\ucsBug{28253})
\item openoffice.org (CVE-2012-1149, CVE-2012-2334, CVE-2012-2665) (\ucsBug{27192})
\end{itemize}

\item Several packages have been converted to use the common shell
library to create their initial log files (\ucsBug{24730}).

\item The copyright files of the Univention packages have been reviewed. Some copyright
files were clarified (\ucsBug{26440}).

\item The UCS 3.0 errata updates 94 to 133 have been integrated (\ucsBug{28147})

\end{itemize}


\section{Univention Installer}
\begin{itemize}

\item The partition module of the installer was completely overhauled. It now uses GUID partition tables (GPT)
  instead of a master boot record (MBR). This results in the following changes: (\ucsBug{23990}, 
  \ucsBug{25065},\ucsBug{26045}, \ucsBug{28263}, \ucsBug{28552})
  \begin{itemize}
  \item The GPT uses 64 bit values and is LBA-only, so the previous
    limitation of 2 terabytes no longer exists (up to 9 zetta bytes
    may be addressed now using up to 127 partitions).
  \item Although GPT is part of the extensible firmware interface
    (EFI), UCS can also boot from GPT disks if the system does not
    support (U)EFI.
  \item Some spare space is reserved for later use at the beginning
    (16 MiB) and at the end (128 MiB) of the disk.
  \item The partition module offers up to 5 different partition types:
    data, swap, LVM, BIOS boot and EFI system. The LVM type is only
    shown if enabled by the user.
  \item All new partitions will be aligned at 1 MiB boundaries. This
    avoids performance issues on harddisks or solid state disks using
    4k sectors.
  \item An existing MBR may be converted to GPT. Please make sure in advance, that other operating systems 
    on that harddisks are able to handle a GPT correctly. The order of partitions may vary 
    during conversion, so the conversion process may influence other operating systems.
  \item If for some reason the old partitioning module is needed, it
    may be activated in the GRUB menu of the DVD under
    \emph{Additional options} \ucsRightArrow \emph{Univention
      Installer MBR Mode (deprecated)}.
  \item The default partition sizes for automatic partitioning have been altered:
    \begin{itemize}
    \item \emph{BIOS boot}: 8 MiB\quad or \emph{EFI system}: 256 MiB
    \item \emph{/boot}: 512 MiB
    \item \emph{swap}: minimum 512 MiB; up to 10 GiB
    \item \emph{/}: at least 4 GiB
    \end{itemize}

  \end{itemize}

\item The amd64 installation DVD now supports the installation on
  systems with the \emph{extensible firmware interface} (EFI). EFI is
  detected automatically by Univention Installer (\ucsBug{28672}) and
  the following changes are made:
  \begin{itemize}
  \item Creation of EFI system partition instead of a BIOS boot
    partition during automatic partitioning
  \item Installation of an EFI-capable Grub configuration
  \item Configuration of a new EFI boot entry for GRUB
  \end{itemize}
  In case of a erroneous detection of EFI, the use of EFI may be
  forced by a new GRUB kernel parameter \ucsCommand{use\_efi=yes/no}.

\item The join credentials are now tested before the installation
  proceeds (\ucsBug{7444}, \ucsBug{29286}).

\item The test for a valid windows domain name in the
  \ucsMenuEntry{Basic settings} dialogue now performs additional
  checks (\ucsBug{26027}).

\item The description of base systems has been improved (\ucsBug{25606}).


\item The installer now ignores deprecated files when copying the script
directory during installation (\ucsBug{25595}).

\item The \ucsFile{/etc/fstab} file is now created using UUIDs instead
  of device names (\ucsBug{17018}).

\item A bug in the CDROM drive detection has been fixed that could lead to
  a traceback if no drive had been found (\ucsBug{28512}).

\item A filesystem check for ext* filesystems is now performed after
  the installation (\ucsBug{25284}).

\item The kernel modules for kernel-based modesetting (KMS) have been
  removed from the installer ramdisk (\ucsBug{28267}).

\item The password of the root user is now hashed with sha-512
  (\ucsBug{28271}).

\item The dialogue to configure the bootloader now ignores device
  mapper devices (\ucsBug{9936}).

\item Network configuration via DHCP is now possible for all system
roles (\ucsBug{26417}).

\item The package \ucsName{python-univention-lib} is now installed
  earlier to prevent package dependency problems (\ucsBug{27962}).

\item Before the domain join, the network is started using the
  networking init script instead of using \ucsCommand{ifup}
  (\ucsBug{27931}).

\item Network interfaces are not  reconfigured during installation
  anymore. This fixes problems with incompletely configured IPv6
  interfaces in IPv6-only setups (\ucsBug{28473}).

\item Various code cleanups have been made in the installer scripts
(\ucsBug{28374}).

\item Existing LVM PV signatures of already existing partitions will
  be now removed, before any new LVM physical volume gets created
  (\ucsBug{18771}).

\item A traceback in the partitioning module has been fixed that could
  be triggered by LVM physical volumes that are not assigned to any
  volume group (\ucsBug{25185}).

\item The handling of custom installer scripts in the installer has been
fixed (\ucsBug{28394}).

\item During startup, the installer now waits for the udev driver
  initialization (\ucsBug{20797}).

\end{itemize}

\subsection{Profile-based installation}
\begin{itemize}
\item The profile-based installation was adapted to the GPT changes:
  Old profiles remain valid, but will continue to use only the
  deprecated MBR mode of the Univention installer. To use the GPT
  mode, small changes have to be done to the profile's partitioning
  variables. Please check the extended installation documentation
  manual for a detailed variable description.
\item The auto partitioning now automatically creates an additional
  BIOS boot partition and a EFI system partition.

\item A bug in the partitioning module concerning LVM partitions has
  been fixed (\ucsBug{13936}).

\item A bootloader record entry in the installation profile is now only
required for systems with more than one hard disk (\ucsBug{9936}).

\item The installation profiles shipped on the installation DVD now
  use GPT as partition table type (\ucsBug{29255}).

\item Fixed a missing confirmation dialog when pressing F12 (\ucsBug{29277}).

\end{itemize}


\section{Upgrade provisions (preup and postup scripts)}

\begin{itemize}
\item The update scripts \ucsName{preup.sh} and \ucsName{postup.sh}
  have been adjusted to UCS 3.1 (\ucsBug{28393}, \ucsBug{29208}, \ucsBug{29212}).

\item Apache and the \ucsUMC{} are no longer restarted in
  \ucsName{postup.sh} (\ucsBug{27925}).

\item \ucsName{univention-legacy-kolab-schema} will be marked as
  manually installed in \ucsName{preup.sh} to avoid the autoremove of
  this package (\ucsBug{28900}).

\item An update to UCS 3.1 is prevented if packages are marked as
  \emph{hold} (\ucsBug{19915}).

\item An update to UCS 3.1 is prevented when the 2.6.32 meta package for
  non-PAE (physical address extension) systems is installed on a system capable of PAE (\ucsBug{29384}).

\end{itemize}





\section{Basic system services}

% \subsection{Boot loader}
% \begin{itemize}
% \end{itemize}

\subsection{Linux kernel and firmware packages}
\begin{itemize}
\item The Linux kernel and associated tools and firmware packages have been updated to version 3.2.30. The meta
  packages have been updated and simplified (\ucsBug{26297}, \ucsBug{27456}, \ucsBug{28157}). The update also fixes
  a number of security issues (CVE-2010-4805,CVE-2011-1083,CVE-2011-1747,CVE-2011-2695,CVE-2011-3347,CVE-2011-4131,
  CVE-2012-2119,CVE-2012-2121,CVE-2012-2136,CVE-2012-2313,CVE-2012-2319,CVE-2012-2375,CVE-2012-2390,CVE-2012-3375,
  CVE-2012-2745,CVE-2012-3400,CVE-2012-3430,CVE-2011-3347,CVE-2012-3412,CVE-2012-3511)
  and a filesystem corruption bug when using sparse files with XFS
  (\ucsBug{26255}). It should be noted that the 486 kernel variant no
  longer supports SMP, you should use the 686 flavour instead.
\item The blktap patch (needed for Xen) was integrated into the 3.2.30 kernel package (\ucsBug{28176}).
\item DKMS has been integrated to ease the build of external kernel modules (\ucsBug{16542}).
\end{itemize}

\subsection{Univention Configuration Registry}

\subsubsection{Changes to templates and modules}
\begin{itemize}
\item Variable descriptions were added for several variables  (\ucsBug{28134}).
\item The template for \ucsFile{/etc/network/interfaces} was converted
  to a multifile template and now supports more complex network
  configurations like bridging, bonding and virtual local area
  networks (\ucsBug{26058}).
\end{itemize}

\subsubsection{Internal changes}
\begin{itemize}
\item \ucsUCR{} variable descriptions for unset variables are now also
  shown when searching for variables (\ucsBug{18254}).
\item Error messages are now redirected to the standard error output
  (\ucsBug{23968}).
\item When multiple patterns match the name of a \ucsUCR{} variable the
  longest match is applied first (\ucsBug{28131}).
\item The \ucsCommand{univention-install-config-registry-info} and
  \ucsCommand{univention-install-service-info} commands now provide
  more detailed error messages (\ucsBug{18507}, \ucsBug{3952}).
\item The \ucsCommand{ldapsearch-wrapper} command now also handles
  tabulator characters correctly (\ucsBug{27842}).
\item Multifile templates now also support calling Python modules
  before and after changes (\ucsBug{28229}).
\item In the header of files generated by a multifile template, the
  list of source template file names is now sorted (\ucsBug{26886}).
\item The tab completion of \ucsName{ucr set} has been changed: The
  value is now put in single quotes to protect it from being evaluated
  (\ucsBug{27573}).
\item The \ucsCommand{ucr randpw} functionality has been removed
  (\ucsBug{28431}).
\item \ucsUCR{} now always uses local diversions of configuration
  files (\ucsBug{28750}).
\item Removing a package contributing a subfile to a multifile
  template leads to the diversion being removed. The function
  \texttt{ConfigHandlers.unregister()} function has been changed to
  only return a set of obsolete handlers instead of a set of all
  affected handlers (\ucsBug{26476}).
\end{itemize}


\subsection{Network interface configuration}
\begin{itemize}
\item Network Manager is no longer supported on servers. The package
  \ucsName{univention-network-manager} can be safely removed after the
  upgrade (\ucsBug{26609}).
\item Host routes are now correctly handled for virtual network
  interfaces (\ucsBug{27198}).
\item The host addresses in \ucsFile{/etc/hosts} are now ordered by
  the network interface name (\ucsBug{19440}).
\item \ucsCommand{univention-ipcalc} and \ucsCommand{univention-ipcalc6} now provide a \emph{-{}-help}
  option. The calculation of reverse DNS zones has been fixed
  (\ucsBug{25626}, \ucsBug{28143}).
\item If the UCS system is configured as a DHCP client, the
  nameservers configured via \ucsUCR{} are used as first nameservers
  (\ucsBug{27939}).
\item The fallback link local address is  now configured as a
  class B network (\ucsBug{28468}).
\item The IP range 169.254.0.x has been removed from the link local
  address range (\ucsBug{28468}).
\ucsName{univention-register-network-address} will correcty refresh
the related PTR DNS record (\ucsBug{27937}).
\item The packages required to configure bridging, bonding and VLANs
  are now installed by default (\ucsBug{29115}).
\end{itemize}


% \subsection{Univention Firewall}
% \begin{itemize}
% \end{itemize}


\section{Domain services}
\subsection{OpenLDAP}
\begin{itemize}
\item OpenLDAP has been updated to version 2.4.31
  (\ucsBug{27992}). This update also fixes a minor security issue in
  two modules, which are not enabled by default (\ucsBug{26473})
\item During the update the group \ucsName{DC Slave Hosts} will be
  added to the group \ucsName{Computers} (\ucsBug{20334}).
\item A segmentation fault in the OpenLDAP overlay module \emph{k5pwd}
  was fixed (\ucsBug{28166}).
\item A description for the \ucsUCRV{ldap/replication/preferredpassword}
  has been added (\ucsBug{18389}).

\item The libnss-ldap/NSCD caching handling for the group resolution has
been replaced with a dump of the LDAP groups and memberships to a
single file (\ucsFile{/var/lib/extrausers/group}) which is used by
nss-extrausers. The dump can be triggered via cron and with a listener
module (\ucsName{nss.py}). The update to UCS 3.1 does these changes
automatically. The old mechanism can be re-configured via UCR, for
example:
\begin{ucsConfigFile}
  ucr set nss/group/cachefile=no
  ucr set nscd/group/enabled=true
  ucr nscd/group/invalidate_cache_on_changes=true
  invoke-rc.d nscd restart
  invoke-rc.d univention-directory-listener restart
\end{ucsConfigFile} (\ucsBug{28361}).

\item The new \ucsUCRV{nss/group/cachefile/check\_member} defines
whether the group members should be re-checked in the LDAP directory
(\ucsBug{28401}).

\end{itemize}

% \subsection{LDAP ACL changes}
% \begin{itemize}
% \end{itemize}

%% \subsection{LDAP schema changes}
%% \begin{itemize}
%% \end{itemize}

\subsection{Listener/Notifier domain replication}
\begin{itemize}
\item A debug message has been changed from debug level WARNING to
  INFO (\ucsBug{26916}).
\item The tool \ucsName{univention-replicate-one} has been added to
  re-replicate one given LDAP object. The tool
  \ucsName{univention-directory-replicate-one} has been removed
  (\ucsBug{27625}).
\item Several memory leaks, out-of-memory conditions and a file
  descriptor leak have been fixed (\ucsBug{27330}, \ucsBug{27729},
  \ucsBug{28417}).
\item The locking used by the Listener and Notifier daemons has been
  changed to be atomic (\ucsBug{27730}, \ucsBug{28417}).
\item The implementation of the extended LDAP filter has been removed
  (\ucsBug{27329}).
\item The reporting of errors in Listener modules has been improved
  (\ucsBug{27376}).
\item The old name \ucsCommand{univention-ldap-listener} is deprecated
  since UCS 2.0. The old compatibility link has been removed
  (\ucsBug{28414}).
\item The handling of the Listener cache path has been fixed
  (\ucsBug{27314}).
\item Various internal implementations details in the Listener daemon
  have been cleaned up (\ucsBug{27315}).

\item On new UCS installations the values for the
\ucsUCRVSA{listener/uniquemember/skip} and
\ucsUCRVSA{listener/uniquemember/skip} will be set to \emph{no}. This
means the listener no longer checks if duplicated entries are set for
uniqueMember or memberUid during the replication (\ucsBug{19491}).

\item LDAP move/rename operations used to be replicated as separate
  delete and add operations. They are now replicated as a rename operation
  (\ucsBug{20605}).
\item The script \ucsCommand{univention-update-memberof} from the
  \ucsName{univention-ldap-overlay-memberof} package  now modifies
  group objects instead of removing and adding the membership
  attributes (\ucsBug{27955}).
\end{itemize}

\subsection{Domain joins of UCS systems}
\begin{itemize}
\item DHCP entries are again created for managed and mobile clients (\ucsBug{28209}).
\item Using a password file for joining UCS systems has been fixed
  (\ucsBug{19430}, \ucsBug{28789}).
\item The error reporting in join script execution has been fixed
  (\ucsBug{27753}).
\item The join of Univention Corporate Client systems to a UCS domain
  controller is now supported (\ucsBug{28570}).
\item \ucsName{univention-management-console-server} and
  \ucsName{univention-management-console-web-server} are now restarted
  during join.
\item \ucsName{univention-join} now compares the UCS versions of the
  joined system and the master domain controller. The version of the
  master domain controller has to be higher or equal to the version of
  the joined system. This test can be disabled via the command line
  paramater \ucsName{-disableVersionCheck} (\ucsBug{25824}).
\item The \ucsUMC{} join module now displays the correct status of join scripts
  (\ucsBug{28067}).
\end{itemize}



\section{Univention Management Console}

\subsection{Univention Management Console web interface}
\begin{itemize}
\item The overall performance of the web interface was improved both in terms
  of memory consumption and speed. Nonetheless, a warning is shown in the
  overview if using an old (slow) browser (IE 8 or FF 3.6)
  (\ucsBug{27432}, \ucsBug{29249}).
\item A global limit for LDAP search results in domain related UMC modules
  has been added in order to avoid long loading times. The exact value can
  be controlled via \ucsUCRV{directory/manager/web/sizelimit}, its
  default value is 2000 (\ucsBug{28248}).
\item Users can add (and remove) their favourite modules to a
  prominent place at the module overview (\ucsBug{28074},
  \ucsBug{28800}, \ucsBug{29236}).
\item If only one module is available, it will be opened directly and
  the overview page will be hidden (\ucsBug{26901}).
\item The \ucsUCRV{umc/web/feedback/mail} configures the email address
  for feedback emails (\ucsBug{27702}).
\item The help dialog can now be replaced (\ucsBug{27627}).
\item A notification is displayed if the UMC page requires a reload
  (\ucsBug{24947}).
\item The ordering of modules and categories at the overview was made
  explicit and can now be customized (\ucsBug{26961}).
\item The containers to be searched in the UMC modules are now ordered
  alphabetically (\ucsBug{28069}).
\item UMC modules and flavours can now be deactivated the
  \ucsUCRV{umc/module/MODULENAME/disabled} and the
  \ucsUCRV{umc/module/MODULENAME/FLAVORNAME/disabled}
  (\ucsBug{26721}).
\item Problems with non-responding sessions for UMC domain management
  modules have been corrected (\ucsBug{28108}).
\item The online documentation link have been added to the documentation links (\ucsBug{29101}).
\end{itemize}

\subsection{Univention Management Console server}
\begin{itemize}
\item The ACL evaluation has been changed to improve the login
  performance of UMC in large environments (\ucsBug{26490}).
\item Fixed error handling of the UMC server for invalid \ucsUCR{}
  requests options (\ucsBug{27875}).
\item A traceback related to a missing type check, which occured in
  rare cases, was fixed (\ucsBug{28102}).
\item Fixed module crashes if the response datatype doesn't
  support the \emph{in} operation (\ucsBug{28182}).
\item The processing of certain GET commands has been fixed
  (\ucsBug{27850}).
\item Fixed various server crashes caused by malformed request data
  (\ucsBug{28140}).
\item A join script for the UMC web server  has been added. During a
join on the command line with \ucsName{univention-join} UMC will now be
restarted (\ucsBug{11925}).
\end{itemize}

\subsection{Univention Management Console / Univention Directory
  Manager modules}

\subsubsection{Basic settings / Univention System Setup}
\begin{itemize}
\item The join process has been corrected to use locally pre-installed
  packages during the installation of server role components
  (\ucsBug{28089}).
\item The Firefox 10 packages (\ucsName{firefox-de},
  \ucsName{firefox-en}) are now installed by default (\ucsBug{27909}).

\item When a software component is added, the warning that
  \ucsName{Samba 3} and \ucsName{Samba 4} should not be combined, will
  only be displayed when a Samba version is newly selected that is not
  already installed (\ucsBug{27949}).

\item When using a long (> 13 characters) hostname, a warning is displayed
  that it should not be used in Samba environments(\ucsBug{27854}).

\item The welcome page only displays help texts for modules that are actually
  used during the setup (\ucsBug{27733}).

\item Problems with non-responding sessions for the UMC module have
  been corrected (\ucsBug{28108}).

\item An empty root password is forbidden when installed locally and
  warned against otherwise (\ucsBug{28155}).

\item A new Python class is now used instead a wrapper around
  \ucsCommand{apt-get} in the software selection.

\item The obsolete hook script
  \ucsName{domainname.post/kerberos\_rename} was removed
  (\ucsBug{28427}).

\item During installation no error will be displayed anymore if the
  \ucsUCRV{system/setup/boot/help} is not set (\ucsBug{27900}).

\item The usability of the widget for specifying the system languages
  was improved (\ucsBug{24388}).

\item A bug in the package selection was fixed (\ucsBug{28198}).

\item The page changes now introduce a slight delay to fix problems in
  incorrect virtualisazion environments (\ucsBug{27734}).

\item A problem concerning the language selection has been fixed
  (\ucsBug{28556}).

\item The memberserver role has been removed from the role selection. It
will be re-added in a errata update for UCS 3.1-0 (\ucsBug{29757}).

\end{itemize}

\subsubsection{Users module}
\begin{itemize}
\item LanMAN hashes are no longer created on new UCS 3.1 installations
  (\ucsBug{27860}).

\item User names with upper cases are now allowed (\ucsBug{25656}).

\item It is no longer allowed to create a user with a name which is
  already used by a group and vice versa. The old behavior can be
  configured by setting the
  \ucsUCRV{directory/manager/user\_group/uniqueness} to \emph{false}
  (\ucsBug{26289}).
\item It is no longer possible to create users with the \emph{Samba}
  option if their chosen primary group does not have the \emph{Samba}
  option as well (\ucsBug{26817}).
\item \ucsName{homeSharePath} and \ucsName{homeShare} can now only be
  set if both values are provided (\ucsBug{28996}).

\item Only users with at least one of the options \ucsName{samba} or
  \ucsName{person} can have a \ucsName{Display name} (\ucsBug{27853}).

\item A new \ucsUCRV{directory/manager/user/primarygroup/update} has
been introduced. If set to \emph{false}, the update of the primary group when
creating users is disabled (\ucsBug{18247}).

\end{itemize}


\subsubsection{Extended attributes}
\begin{itemize}
\item Changes to extended attributes immediately alter other modules without
  the need to restart the \ucsUMC{} server (\ucsBug{26677}).
\end{itemize}


\subsubsection{License module}
\begin{itemize}
\item Support for UCS license version 2 has been added
(\ucsBug{28186}, \ucsBug{29079}).

\item The detection of invalid license files has been improved
  (\ucsBug{26047}).

\item Fixed a bug, which could result in the deletion of arbitrary
  files (\ucsBug{28189}).

\item The import of a free-for-personal-use license is now possible
  (\ucsBug{29051}).

\item The license handling code for Univention Groupware Server and
  the Open-Xchange Appliance Edition has been removed (\ucsBug{26919}).

\item The users \emph{http-hostname}, \emph{http-proxy-hostname} and
  \emph{zarafa-hostname} are no longer be accounted to the
  license. Also, some accounts typically present on a Microsoft Small
  Business Server, are no longer accounted to the license (\ucsBug{28930}).

\item A bug in UCS license version 1 regarding the counting of system
  accounts has been fixed (\ucsBug{28944}).
\end{itemize}


\subsubsection{Mail module}
\begin{itemize}
\item New syntax classes have been implemented for shared folder ACLs,
  which prevent performance problems in large environments
  (\ucsBug{28084}).
\item The mail module is now also available on systems with only a UCS
  license (i.e. without a UGS sublicense) (\ucsBug{27985});
\item An LDAP search syntax for mail home servers has been added
  \ucsBug{27820}).
\item The attribute \ucsName{Mail home server} (present in the user
  and the mail folder module) is now correctly set (\ucsBug{26425}).


\end{itemize}

\subsubsection{Services module}
\begin{itemize}
\item The search for service names has been fixed (\ucsBug{28132}).
\item The descriptions of the services are now correctly localized
  (\ucsBug{27742}).
\end{itemize}


\subsubsection{Software management}
\begin{itemize}
\item The module App Center has been developed and added to UMC. It allows
  users to easily install new applications on an UCS system
  (\ucsBug{27458}, \ucsBug{28002}, \ucsBug{28599}, \ucsBug{28798},
  \ucsBug{28940}, \ucsBug{28718},  \ucsBug{29035}, \ucsBug{29039},
  \ucsBug{29036}, \ucsBug{29081}, \ucsBug{28706}, \ucsBug{29037},
  \ucsBug{28740}, \ucsBug{28793}, \ucsBug{27998}, \ucsBug{29246},
  \ucsBug{29243}, \ucsBug{29339}).

\item The \ucsName{univention-management-console-module-packages}
  module has been renamed to
  \ucsName{univention-management-console-module-appcenter}.
  (\ucsBug{28626}).

\item The module for the installation of additional software packages
  was reworked (e.g. added possibility to install/remove multiple
  packages at once) (\ucsBug{27997}, \ucsBug{28088}, \ucsBug{17337}, \ucsBug{28010}).

\item The configuration of components and the repository now occurs
  through the Univention App Center (\ucsBug{26784}).
\end{itemize}

\subsubsection{Online update module}
\begin{itemize}
\item The errata level is now also displayed along with the version (\ucsBug{26906}).
\item The page refresh after updates has been improved (\ucsBug{27460}).
\item After package upgrades it is now checked again, whether new
  updates are available (\ucsBug{27794}).
\item Display a strong warning to not power off the system during update
  (\ucsBug{27899}).
\item A link to the errata web site containing the available updates
  is now displayed (\ucsBug{25597}).
\item Only the last 2500 lines of a log file are now shown (\ucsBug{26279}).
\item Do not rely on \ucsUCRV{update/available} alone when checking
  for updates in easy mode (\ucsBug{26278}).
\end{itemize}


\subsubsection{Computers module}
\begin{itemize}

\item The IP address will now be shown in the search result list if
  the IP address was used as filter (\ucsBug{25242}).

\item An \ucsUDM{} module for arbitrary Linux computers has been added
  (\ucsBug{27587}).

\item Fixed a problem when removing the last IP address of a computer account
  (\ucsBug{28622}).

\item When the DNS forward zone entries of a computer object are
  modified, the related PTR records are now updated as well(\ucsBug{26307}).

\item Corrected an error which occured when creating a computer object
  with a DNS forward zone differing from the kerberos domain
  (\ucsBug{25172}).

\end{itemize}

\subsubsection{Policies}
\begin{itemize}
\item Fixed a caching problem that displayed incorrectly referenced objects for
  some policies (\ucsBug{25638}).

\item Every computer module stored under
  \ucsName{/usr/share/pyshared/univention/admin/handlers/computers}
  is now automatically imported (\ucsBug{28461}).

\item The descriptions of certain attributes in package management policies have
  been corrected (\ucsBug{18200}).

\item A bug was fixed, which prevented changing the user's password if
  a mail quota policy was fixed (\ucsBug{28430}).

\item Some special characters (e.g. umlauts) can no longer be used as
  policy names (\ucsBug{28098}).

\item The \ucsName{nouveau} driver has been added to the list of
  available X servers in the \ucsName{Display} policy (\ucsBug{23124}).

\item The \ucsUCR{} policy now also applies to Univention Corporate
  Client systems (\ucsBug{28852}).

\end{itemize}

\subsubsection{Printers module}
\begin{itemize}
\item An incorrect error message when trying to add printers to a
  printer group with quota support has been fixed (\ucsBug{19627}).
\item A new option \ucsName{Use Windows client driver} has been added.
  It is enabled by default and allows Windows clients to print using
  locally installed drivers (\ucsBug{27340}).
\end{itemize}

\subsubsection{DHCP module}
\begin{itemize}
\item When all DHCP range entries are removed from a DHCP subnet, the
  related LDAP attribute \ucsName{dhcpRange} will be deleted from the
  LDAP object (\ucsBug{28623}).
\end{itemize}

\subsubsection{Other modules}
\begin{itemize}
\item The page layout of the samba domain page has been improved (\ucsBug{27267}).

\item Alternative email adresses can now be configured for mail groups
  (\ucsBug{27871}).

\item The LDAP filter for \ucsName{container/cn} and
  \ucsName{container/ou} no longer matches the LDAP base object
  (\ucsBug{25737}).

\item Various UDM objects now use a new default search: It searches in
  several important attributes (specific for every class of object) at
  once (\ucsBug{24341}).

\item An error message in the processing of incorrect syntax
  definitions has been clarified (\ucsBug{28408}).

\item Fixed an error when retrieving possible choices for a syntax while the
  underlying LDAP extension has not yet been installed (\ucsBug{29333}).

\item Visual glitches in the display of UDM detail pages are now
  prevented (\ucsBug{28400}).

\item The escaping of characters specific to LDAP search filters has
  been improved (\ucsBug{16387}).

\item A UDM module to create Kerberos KDC entries has been added
  (\ucsBug{28379}).

\item The \ucsName{Networks} module no longer suggests IP addresses
  which are already in use (\ucsBug{28137}).

\item Legacy policies and \ucsUDM{} modules for the management of UCD and UCS
TCS have been moved into the package
\ucsName{python-univention-directory-manager-legacy-ucd-tcs}. This
package will be automatically installed during the update to UCS 3.1 
(\ucsBug{27617}).

\item The module \ucsName{Navigation} have been renamed to
  \ucsName{LDAP directory} (\ucsBug{26511}).

\item The caption of the \ucsMenuEntry{UMC Command List} has been
  fixed (\ucsBug{26705}).

\item Various spelling mistakes in several modules have been corrected
  (\ucsBug{28397}).

\end{itemize}

\subsection{Univention Directory Manager command line interface and related tools}
\begin{itemize}
\item \ucsCommand{proof\_uniqueMembers} has been extended to also
  remove users which no longer exist (\ucsBug{27929}). The
  \emph{uniqueMember} and \emph{memberUid} attributes are now compared
  case-insensitive (\ucsBug{21026}).

\item \ucsCommand{univention-sync-memberuid} has been rewritten
  (\ucsBug{27907}).

\item \ucsCommand{univention-dnsedit} now provides more help
  information (\ucsBug{27864}).

\item It is now possible to pass
  \ucsCommand{univention-directory-manager} the password for the LDAP
  connection via a file by using the option \texttt{-{}-bindpwdfile}
  (\ucsBug{19978}).

\item The \ucsUCRV{directory/manager/cmd/sockettimeout}
  can be used to configure the time interval (in seconds) the CLI client
  waits for a socket file of the CLI server

\item \ucsCommand{convert-user-base64-photos} now displays an error
  message if the LDAP server is unreachable (\ucsBug{27961}).

\item The deprecated log file \ucsFile{admin-cmd.log} has been
  replaced with the current \ucsFile{directory-manager-cmd.log} in
  several commandline tools (\ucsBug{13105}).

\item The postinst script of
\ucsName{univention-directory-manager-modules} no longer calls the update script
\ucsName{convert-user-base64-photos} for new installations
(\ucsBug{28366}).

\item A traceback when trying to remove non-existent objects has been fixed (\ucsBug{27956}).

\end{itemize}


\subsection{Development of modules for Univention Management Console}
\begin{itemize}
\item Several decorator functions have been added to help developers to write
  and extend UMC modules. These functions are documented in
  \ucsName{univention-management-console-doc} (\ucsBug{27716}).
\item Arguments passed to UMC modules can now be validated and sanitized
  uniformly. The mechanism is extendable and is documented in
  \ucsName{univention-management-console-doc} (\ucsBug{27720}).
\item Added a new widget to handle cases where a huge amount of entries
  would be returned possibly blocking the form for several minutes. All
  entries can be searched remotely from within this widget (\ucsBug{26556}).
\item Fixed dependent widgets not updating initially when the other widget was
  initialized with an empty value (\ucsBug{29333}).
\item The prefix of all the package management UMCP commands have been
  renamed from \emph{packages/*} to \emph{appcenter/*}.
  Corresponding \ucsName{UMC}-policies have to be adapted manually
  (\ucsBug{28626}).
\item Added a function to set further details of a module's title. This also
  fixes some possible escaping problems (\ucsBug{28146}).
\item The MultiObjectSelect widget formatter now supports strings (\ucsBug{26780}).
\item The initialization of static values in the Form widget was fixed
  (\ucsBug{26780}).
\item The Dojo Toolkit library has been updated to version 1.8.1, and the UMC
  framework as well as all UMC modules have been ported and adapted to Dojo
  API changes (\ucsBug{26857}, \ucsBug{26826}, \ucsBug{28498},
  \ucsBug{28551}, \ucsBug{28516}, \ucsBug{28811}, \ucsBug{28710},
  \ucsBug{27397}, \ucsBug{28796}, \ucsBug{29087}, \ucsBug{28018},
  \ucsBug{29224}, \ucsBug{29232}, \ucsBug{29382}, \ucsBug{29390},
  \ucsBug{29250}, \ucsBug{29238}).
\item The PasswordInputBox widget has been extended with a function to
  reset the whole widget (\ucsBug{27135}).
\item The \emph{confirmForm} function was added to
  \ucsName{umc.dialog} (\ucsBug{26791}).
\item Visibility issues with doubled entries in grids during opening a
  module have been corrected (\ucsBug{25476}).
\item The package
  \ucsName{univention-manangement-console-frontend-src} has been added
  which contains the uncompressed javascript source files
  (\ucsBug{28018}).
\item Problems with dependencies of form items have been resolved, an
  additional method for handling of asynchronous data loading has been
  added (\ucsBug{26214}, \ucsBug{28847}).
\end{itemize}




\section{Software deployment}

\begin{itemize}
\item The handling for errata updates has been changed in UCS 3.1. One
errata scope will be released for every UCS patchlevel release
(\ucsBug{26387}).

\item The upgrade maintenance check did not recognize new package
policies. This issue has been fixed (\ucsBug{27171}).

\end{itemize}

\subsection{Repository handling}
\begin{itemize}
\item \ucsCommand{apt-mirror} has been patched to honor the limit of
  forked \ucsCommand{apt-ftparchive} and \ucsCommand{gzip} processes
  as configured by the \ucsUCRV{repository/mirror/threads}
  (\ucsBug{25518}).
\item \ucsCommand{univention-repository-update} now supports the
  \emph{--iso} option to update a local repository from a downloaded
  \emph{.iso} file (\ucsBug{26239}).

\item The errata updates for components will now be mirrored into the
  local repository (\ucsBug{28187}).

\item The symlinks in the \ucsFile{isolinux} directory of the repository will no
  longer be created (\ucsBug{17835}).

\item The PXE symlinks for the Univention Net Installer kernel and the
  initrd are now created by \ucsName{univention-repository-create}
  (\ucsBug{24777}).

\item If the \ucsUCRV{repository/mirror/recreate\_packages} is set to \emph{no}, the
packages files from the reporitory mirror will be used (\ucsBug{23346}).

\item The obsolete function
  \ucsCommand{errata\_component\_update\_temporary\_sources\_list()}
  has been removed (\ucsBug{27152}). Some old code still using
  deprecated functions has been cleaned up (\ucsBug{27745}).
\end{itemize}



\subsection{Software deployment command line tools}
\begin{itemize}
\item The new tool \ucsCommand{univention-remove} can be used to
  uninstall packages (\ucsBug{19468}).
\end{itemize}


\subsection{Software monitor (univention-pkgdb)}
\begin{itemize}
\item The \ucsName{univention-pkgdb} has been cleaned up for speed,
  code, and usability improvements (\ucsBug{13538}, \ucsBug{21873},
  \ucsBug{19467}, \ucsBug{29196}).

\item The \ucsUCRV{pgsql/pkgdb/networks} now contains a list
  (separated by spaces) of PostgreSQL CIDR-ADDRESS definitions of the
  networks that are allowed to access the \ucsName{univention-pkgdb}
  database. By default the access is not restricted. The
  \ucsUCR{} variables \ucsUCRVSA{pgsql/pkgdb/network} and \ucsUCRVSA{pkgsql/pkgdb/netmask} are
  not supported anymore and should be migrated to the new
  \ucsUCRV{pgsql/pkgdb/networks} if required (\ucsBug{27873}).

\item The description of the \ucsUCRV{pkgdb/scan} has been improved
  (\ucsBug{22578}).

\item \ucsName{univention-pkgdb} now stores the system architecture in
  the database (\ucsBug{11364}).

\item It is now possible to remove a system from the database using
  \ucsName{univention-pkgdb-scan -{}-remove-system} (\ucsBug{14929}).

\item The \ucsUCRV{pkgdb/requiressl} is set to \ucsName{true} during the
  update of \ucsName{univention-pkgdb-tools}. If this \ucsUCRV{} is
  \emph{true}, clients use only SSL secured
  connections to the server. If this variable is set to \emph{true},
  software monitor servers only allow SSL secured
  connections to the server. It can be set to \ucsName{false} to
  restore the old behaviour, if required for compatibility with older
  \ucsUCS{} versions (\ucsBug{27873}).

\item The join script now waits for up to a minute for the service
  record to appear before starting the initial scan of the system
  (\ucsBug{13890}).

\item The \ucsName{--clean} action is not necessary anymore. Thus the
  corresponding weekly cronjob univention-pkgdb-clean has been removed
  (\ucsBug{29266}).
\end{itemize}


\section{Univention Library}
\begin{itemize}
\item \ucsName{get\_default\_netmask()} from \ucsFile{base.sh} will now return a
  netmask if the default address is IPv4 and a prefix length if it is IPv6
  (\ucsBug{26216}).
\end{itemize}


\section{System services}

\subsection{DHCP}
\begin{itemize}
\item The remaining DHCP configuration files were moved from
  \ucsFile{/etc/dhcp3/} to \ucsFile{/etc/dhcp/}, which is the new
  directory since UCS 3.0 (\ucsBug{27829}).

\item \ucsFile{/etc/init.d/isc-dhcp-server} now prints a message that
  the script is over-ruled by \ucsFile{/etc/init.d/univention-dhcp}
  (\ucsBug{17802}).

\item The permission bit of \ucsFile{/etc/init.d/isc-dhcp-server} has been
  fixed to be executable (\ucsBug{25377}).

\item Several statements, which are illegal inside a pool section, are
  no longer added to the dhcpd configuration (\ucsBug{7585}).
\end{itemize}

\subsection{DNS}
\begin{itemize}
\item \ucsFile{/etc/init.d/bind9} now checks the autostart setting of
  \ucsName{univention-bind} instead of \ucsName{univention-bind-proxy}
  (\ucsBug{23811}).
\item The \ucsUCRV{dns/allow/transfer} has been added to make the
  \ucsName{allow-transfer} option of \ucsName{bind9} configurable. It
  and the \ucsUCRV{dns/allow/query} are initialized to \texttt{any}
  (only supported by the LDAP backend) (\ucsBug{10365}).
\item The new file \ucsFile{local-predlz.conf.samba4} gets included in
  \ucsFile{named.conf.samba4} before the declaration of the Samba 4
  DLZ zone (\ucsBug{26302}).
\item A newly joined UCS 3.1 DNS server will be automatically appended
  to the list of the name servers in the DNS forward and reverse zone
  (\ucsBug{8036}).
\item A potential crash through incorrect memory handling (double
  free) has been fixed in the LDAP backend of bind9 (\ucsBug{11295}).
\item In case \emph{samba4} is configured in \ucsUCRV{dns/backend}
  the name server is reloaded after a DNS zone was added or removed
  (\ucsBug{29289}).
\item The network scripts \ucsName{ifup} and \ucsName{ifdown} now
  restart bind instead of using \ucsName{rndc reconfig}
  (\ucsBug{29659}).
\end{itemize}

\subsection{Cyrus}
\begin{itemize}
\item The setting of the default Cyrus ACLs has been fixed (\ucsBug{17994}).

\item A default (7389) for the \ucsUCRV{ldap/server/port} has been added
in the \ucsUCR{} templates of \ucsName{univention-mail-cyrus} and
\ucsName{univention-mail-cyrus-murder} (\ucsBug{27305}).

\item Creation of the default Sieve script in
  \ucsName{univention-cyrus-mkdir} has been fixed (\ucsBug{17928}).
\end{itemize}

\subsection{Postfix}
\begin{itemize}
\item A default (7389) for the \ucsUCRV{ldap/server/port} has been
  added in the \ucsUCR{} templates of
  \ucsName{univention-mail-postfix} (\ucsBug{27305}).
\end{itemize}

% \subsection{Spam detection and countermeasures}
% \begin{itemize}
% \end{itemize}

\subsection{Printing services}
\begin{itemize}
\item The run time directory for new installations of the PDF printer
  package was changed from \ucsFile{/var/cache/cups-pdf/} to
  \ucsFile{/var/spool/cups-pdf/} (\ucsBug{17548}). An old override for
  file permissions of \ucsCommand{cups-pdf} has been removed
  (\ucsBug{17547}).

\item The package \ucsName{univention-foomatic-ppds} has been split
  from \ucsName{univention-printserver-pdf} into its own source
  package (\ucsBug{28494}).

\item The test for disabled printers in
  \ucsName{univention-check-printers} has been fixed (\ucsBug{26767}).

\item A python error in \ucsName{cups-printers.py} has been fixed
  (\ucsBug{29674}).
\end{itemize}

\subsection{Kerberos}
\begin{itemize}
\item The Heimdal KDC daemon is now restarted in the Heimdal join
  script (\ucsBug{25103}).
\end{itemize}

\subsection{Proxy services}
\begin{itemize}
\item The \ucsFile{squid.conf} template has been cleaned up (\ucsBug{27815}).

\item The Dansguardian virus check can now be disabled via the
  \ucsUCRV{squid/virusscan}, the content scan via the
  \ucsUCRV{squid/contentscan}. If both variables are set to
  \emph{false}, Dansguardian is disabled and no longer used by the
  Squid proxy as webfilter backend (\ucsBug{13674})
\end{itemize}

% \subsection{Apache}
% \begin{itemize}
% \end{itemize}

\subsection{Nagios}
\begin{itemize}
\item Some warnings in the UNIVENTION\_JOINSTATUS plugin have been
  fixed (\ucsBug{18828}).
\end{itemize}


\subsection{SSL}
\begin{itemize}
\item \ucsName{univention-ssl} now also stores the validity for the root
certificate. The expiration date of the host certificate is
automatically stored in the new \ucsUCRV{ssl/validity/host}, the
expiration date of the root certificate in the new
\ucsUCRV{ssl/validity/root}. The old \ucsUCRV{ssl/validity/days} will be
deleted during the update. The UMC and the nagios test UNIVENTION\_SSL
now examine both the host and the root certificate (\ucsBug{27002},
\ucsBug{25788}).

\item SSL certificates are now also created for Univention Corporate
  Client systems (\ucsBug{28571}).

\item \ucsName{univention-ssl} now sets a default (true) for the
 \ucsUCRV{ssl/validity/check} on all system roles (\ucsBug{19736}).

\item A path handling problem was fixed in \ucsName{univention-ssl}
  (\ucsBug{26572}).
\end{itemize}

\subsection{NFS}
\begin{itemize}
\item The join script for \ucsName{univention-nfs-server} is now
  executed after installation (\ucsBug{24086}).

\item NFS version 4 for the NFS server has been disabled by
  default. It can be configured by setting the
  \ucsUCRV{nfs/nfsd/nfs4} (\ucsBug{28500}).

\item The \ucsUCR{} template for
  \ucsFile{/etc/modprobe.d/nfs-kernel-lockd} has been updated to the
  new modprobe naming scheme (\ucsBug{27959}).

\item The handling of LDAP exceptions in the policy script
  \ucsName{nfsmounts} has been improved (\ucsBug{28194}).

\item The new listener module \ucsName{nfs-homes} from the package
  \ucsName{univention-nfs-server} is responsible for the creation of
  the user's \ucsName{homeSharePath} on NFS home shares. This feature can be
  disabled by setting the \ucsUCRV{nfs/create/homesharepath} to
  \emph{false} and restarting the Univention Directory Listener
  ((\ucsBug{25375})).
\end{itemize}


\subsection{Other services}
\begin{itemize}
\item With the update to UCS 3.1 the \ucsUCRV{ntp/signed} will be set
  to \emph{yes} if the variable was not set before. This will enable signed
  NTP packages (\ucsBug{26223}).

\item The init script of \ucsName{univention-runit} is now stopped later. This
  prevents early terminations of services like \ucsName{libvirt-bin},
  which otherwise fails to save virtual machines (\ucsBug{28817}).

\item The PAM configuration in \ucsName{univention-postgresql} is now
  updated in a join script to ensure it contains the current
  \ucsFile{machine.secret} (\ucsBug{28795}).
\end{itemize}


\section{Virtualisation}

\subsection{libvirt}
\begin{itemize}
\item Libvirt has been updated to version 0.9.12
  (\ucsBug{27612}).
\item The init script
  \ucsCommand{/etc/init.d/univention-virtual-machine-manager-node-common}
  has been deprecated in favour of
  \ucsCommand{/etc/init.d/libvirt-bin}. The order in which the
  virtualisation init scripts are called, has been fixed (\ucsBug{19145}).

\item Broken qcow2 files missing their backing store no longer
  prohibit the storage pool form starting (\ucsBug{22098}).
\end{itemize}

\subsection{Univention Virtual Machine Manager}
\begin{itemize}
\item Some spelling mistakes have been corrected (\ucsBug{24831}).
\item The \ucsUVMM{ daemon} was changed to not log libvirt-related messages to
  \ucsFile{/var/log/univention/virtual-machine-manager-daemon-errors.log},
  since they are also logged to
  \ucsFile{/var/log/univention/virtual-machine-manager-daemon.log}
  (\ucsBug{24135}).

\item The logging level of the \ucsUVMM{ daemon} can now be controlled
  by the \ucsUCRV{uvmm/debug}, \ucsUCRVSA{uvmm/debug/command},
  \ucsUCRVSA{uvmm/debug/ldap}, and
  \ucsUCRVSA{uvmm/debug/unix}. Accepted values are \emph{DEBUG},
  \emph{INFO}, \emph{WARNING}, \emph{ERROR}, \emph{CRITICAL}
  (\ucsBug{28136}.

\item The \ucsUCRV{uvmm/umc/vnc/host} controls the format of the VNC
  link: By default the IPv4 address is used, which can also be
  explicitly set to \emph{IPv4}. \emph{IPv6} explicitly requests an
  IPv6 address. To both values a Python regular expression can be
  appended by a single blank, which only accepts matching addresses as
  returned by \emph{getaddrinfo()}. \emph{NAME} passes on the DNS name
  of the virtualization server unmodified, but substitutions can be
  performed by appending multiple \emph{old=new} pairs separated by
  spaces, which are applied to the name in order. Any other values
  overwrites the individual server name by a fixed static string
  (\ucsBug{28104}).

\item The escaping of XML special characters has been fixed;
  \texttt{\&} and \texttt{<} are prohibited in snapshot names
  (\ucsBug{27189}).

\item Profiles for Microsoft Windows 2012 and Microsoft Windows 8 have
  been added (\ucsBug{28420}). Profiles for UCS-3.1 have been added
  while profiles for UCS-2.4 are no longer created on new
  installations (\ucsBug{29338}).

\item The configuration file for the loopback devices has been renamed
  (\ucsBug{14824}).

\item The file format used by image files can now be changed when
  attaching existing images to virtual machines. This also fixes a bug
  when \emph{qcow2} images were wrongly attached as
  \emph{raw}-images. (\ucsBug{25169}).

\item The cache strategy for block devices can be configured now. The
  default for harddisks was changed to \emph{none} (\ucsBug{23445}).

\item Snapshots are now deleted when a virtual machine is removed
  (\ucsBug{25181}).

\item Migration now waits until the target domain is running again
  (\ucsBug{25857}).

\item Invalid XML data in storage volume and domain information is now
  silently ignored (\ucsBug{26681}).

\item The \emph{Change medium}-dialog will now show the available
  mediums (\ucsBug{27817}).

\item Editing of VMs which are not cleanly shutdown is disabled (\ucsBug{25467}).

\item Now the initial values will be loaded properly when editing an
  existing interface (\ucsBug{29274}).
\item An internal bug was fixed, which caused UVMM to hang when no profiles were found (\ucsBug{29375}).
\end{itemize}

\subsection{Xen}
\begin{itemize}
\item The \ucsName{Xen} hypervisor and tool stack have been updated to
  version 4.1.3 (\ucsBug{27021}).

\item The signed build of the GPLPV windows drivers have been updated
  to version 0.11.0.369 (\ucsBug{26613}).

\item The order in which the Xen init scripts are called, has been
  fixed, thus correctly suspending and resuming guest machines on host reboot (\ucsBug{28813}).
\end{itemize}

\subsection{QEMU/kvm}
\begin{itemize}
\item The \ucsName{QEMU-kvm} hypervisor and tool stack have been
  updated to version 1.1.2 (\ucsBug{28282}, \ucsBug{28283}).
\item \ucsName{QEMU-kvm} switched from the deprecated
  \ucsName{etherboot-qemu} to \ucsName{ipxe} to provide the network
  card ROM images. Old snapshots and suspended virtual machines still
  need the old ROM images. \ucsName{QEMU-kvm} has been patched to use
  the old \ucsName{etherboot-qemu} ROMs for \emph{pc-0.14} and older
  VMs (\ucsBug{24702}).
\end{itemize}


\section{Desktop packages}
\begin{itemize}

\item Via the new \ucsUCRV{gdm/systemmenu} it is possible to allow or
disallow the shutdown or reboot of the system via the login manager
\ucsName{gdm} or the desktop environment \ucsName{KDE} (default is
true). The \ucsUCRV{gdm/menu/system} controls whether the login manager
provides buttons for the shutdown and restart of the computer (default
is \emph{false}). By default, the system can now be turned off or rebooted via
the desktop environment but not via the login manager (\ucsBug{25677}).

\item The previously used form of internationalization
in firefox with language packs is no longer supported. Instead, there
are now two separate packages, \ucsName{firefox-en} for the English
and \ucsName{firefox-de} for the German version of Firefox (\ucsBug{28006}, \ucsBug{27986}).

\item An entry \ucsName{Server Administration} (link to the
UCS overview page) has been added to the KDE menu (\ucsBug{28586}).

\item The NSS LDAP configuration has been fixed: It is now
possible for the DBus daemon to resolve usernames (\ucsBug{28807}).

\item The download url in \ucsName{univention-flashplugin} has been updated (\ucsBug{28855}).
\end{itemize}


\section{Services for Windows}

\begin{itemize}
\item Support for Microsoft Windows 8 has been added (\ucsBug{28015}).
\item The \ucsUDL{} module for generating Samba shares now removes the
  local share configuration if the share host has been changed to a
  different host. The module now also uses the current fully qualified
  domain name and IP (\ucsBug{27253}).
\item The script \ucsCommand{remove\_sambapwdlastset} has been renamed to
  \ucsCommand{remove\_sambapwdmustchange} (\ucsBug{27838}).
\end{itemize}

\subsection{Samba 3}
\begin{itemize}
\item Samba has been updated to version 3.6.8 (\ucsBug{24231}):
	\begin{itemize}
	\item The \ucsUCRV{samba/winbind/max/clients} has been introduced
      with a default value of 500.
	\item Default value for \texttt{use spnego} is not set explicitly
      any longer to avoid a deprecation warning.
	\item New \ucsUCRV{samba/vfs/acl\_xattr/ignore\_system\_acls}.
	\item New \ucsUCRV{samba/max\_protocol}.
	\item Removed support for legacy \ucsUCRV{samba/idmap/domains}.
	\end{itemize}
\item Winbind is now installed on all Samba 3 systems (\ucsBug{26200}).
\item For new UCS installations the default for the minimal password
  length in the Samba domain policy has been increased from 5 to 8 characters
  to match the standard UCS password policy default (\ucsBug{21262}).
\item A traceback has been fixed in the \ucsCommand{kerberize\_user} and
  \ucsCommand{remove-samba-account} scripts (\ucsBug{27841}).
\item The Logrotate configuration for the main Winbind log files has
  been extended (\ucsBug{4738}).
\item A minor traceback during the installation of
  \ucsName{univention-samba-local-config} has been fixed
  (\ucsBug{27887}).
\item Inheritance of group ownership from parent directories by means
  of the set group id (SGID) bit is now honoured during file and
  directory creation via SMB (\ucsBug{27950}).
\item During the first join of an UCS member server into a Samba 4
  domain, the \ucsUCRV{samba/share/home} will be set to \emph{yes}, to
  provide access to the home directories via samba share
  (\ucsBug{25339}).
\end{itemize}

\subsection{Samba 4}
\begin{itemize}
\item Samba 4 has been updated to version 4.0 RC6 (\ucsBug{27457},
  \ucsBug{24247}, \ucsBug{29529}).
\item Periodic machine password changes are now possible on Samba 4
  domain controllers. It is enabled automatically during update by
  setting the \ucsUCRV{server/password/change} to \emph{yes}
  (\ucsBug{24703}).
\item Two trivial tracebacks in
  \ucsCommand{univention-samba4-site-tool.py} have been fixed
  (\ucsBug{27153}).
\item The \ucsName{samba4} init script now ensures that all Samba
  processes terminate when \emph{stop} is called (\ucsBug{27132}).
\item The password complexity check has been fixed for UTF-8 characters
  (\ucsBug{25554}).
\item \ucsCommand{univention-s4search} has been fixed to use the
  machine account for authorization by default (\ucsBug{28279}).
\item A bug affecting rejoins of Samba 4 domain controllers has been fixed, where the
  RID pool rIDNextRID counter was lost, possibly blocking the local
  creation of new Samba 4 accounts. Now this counter will be preserved
  during rejoins (\ucsBug{28373}).
\item Error handling and performance of the Samba 4 join and provision
  scripts have been improved (\ucsBug{26731}).
\item When joining a domain the GC record registration is now
  controlled by the \ucsUCRV{dns/register/srv\_records/gc} and the PDC
  record registration by the \ucsUCRV{dns/register/srv\_records/pdc}
  (\ucsBug{28065}).
\item If Samba 4 cannot itself find a DC during a join, the
  \ucsName{univention-samba4} joinscript will try joining against all
  DCs it an find in LDAP offering Samba 4 services in turn. The
  \ucsUCRV{samba4/dc} can be used to specify a server to join against
  instead. Kerberos will not be used during Samba domain join
  (\ucsBug{27469}).
\item SYSVOL NTACLs and fACLs were set to the UCS 3.0 defaults on
  every run of the Samba 4 join script. This has been fixed
  (\ucsBug{28894}).
\item The handling of LDAP rename operations has been improved in the
  \ucsName{samba4-idmap} listener module (\ucsBug{28643}).
\item The source of two tracebacks has been fixed in the
  \ucsName{samba4-idmap} listener module (\ucsBug{26013}).
\item Inheritance of group ownership from parent directories by means
  of the set group id (SGID) bit is now honoured during file and
  directory creation via SMB (\ucsBug{28711}).
\item Samba 4 now uses its own version of Heimdal Kerberos (instead
  of using the system-wide Heimdal libraries) (\ucsBug{29005}).
\item The post-installation script of the \ucsName{univention-samba4}
  package was improved to avoid a deadlock due to a leftover Samba
  process (\ucsBug{28583}).
\item The join script of \ucsName{univention-samba4} has been improved
  to backup and merge the Samba 4 machine keytab in case of a re-join
  (\ucsBug{25393}).
\item Group SIDs appearing in the owner field of a NT security
  descriptor are now supported in NTACLs, e.g. of GPO files in the
  sysvol share. To avoid privilege escalation in cases of overlapping
  uid and gid numbers, the Unix file ownership is mapped to 0 for this
  kind of NTACLs (\ucsBug{28737}).
\item A new script \ucsCommand{univention-ad-takeover} has been added
  to support the takeover of users and computers from an Active
  Directory server (\ucsBug{28390}).
\item The reliability of DNS service principal creation was improved
  (\ucsBug{29217}).
\item The first system that provides Samba 4 will remove systems
  without Samba 4 from the Kerberos DNS service records. This
  avoids Kerberos errors in environments where Samba 4 is run
  on a DC Backup but not on a DC Master (\ucsBug{29225}).
\end{itemize}

\subsection{Univention S4 Connector}
\begin{itemize}
\item The domain password setting attributes are now synced between UCS and
Samba 4. During the update to UCS 3.1 the password settings in UCS will be
overwritten with the Samba 4 settings (\ucsBug{24237}).
\item If the connection to the UCS LDAP server was closed, the connection
will be reopened and the last search will be started again (\ucsBug{27176}).
\item The S4 Connector will be restarted at the end of the join process
so the UCR values are re-read (\ucsBug{26859}).
\item The debug message if SSL is disabled has been improved
(\ucsBug{25835}).
\item The init script has been made more resilient (\ucsBug{27175}).
\item The attribute \emph{mailPrimaryAddress} is now synced to the
  Samba 4 attribute \emph{mail} instead of the UCS attribute \emph{mail}.
\item A traceback has been fixed which occured if the LDAP and Samba 4
  base DNs are not equal (\ucsBug{28573}).
\item The group membership cache has been improved (\ucsBug{28689}).
\item A bug in the creation of new DNS zones in UCS has been fixed
  (\ucsBug{28889}).
\item A traceback caused by an special/invalid Kerberos keytypes was
  fixed. This keytype is typically issued by Windows 2008 AD servers
  (\ucsBug{28907}, \ucsBug{28932}).
\item The mapping has been enhanced to allow group name mappings,
  e.g. to abstract from locale specific names in Samba 4
  (\ucsBug{28910}).
\item The extended attributes will now be loaded before the first
  object will be created (\ucsBug{28981}).
\item A write-lock problem in the \ucsName{univention-ad-takeover} script has been fixed (\ucsBug{29751}).
\end{itemize}

\subsection{Univention Active Directory Connector}
\begin{itemize}
\item The Univention Active Directory Connector supports now Microsoft
  Server 2012 (\ucsBug{28013}).
\item The debug message if SSL is disabled has been improved
  (\ucsBug{25835}).
\item \ucsUCR{} variables have been added to handle which objects will
  be ignored by the Active Directory connector
  (\ucsUCRVSA{connector/ad/mapping/*/ignorelist}) (\ucsBug{20517}).
\item The init script has been made more resilient (\ucsBug{27175}).
\item The attribute \emph{mailPrimaryAddress} is now synced to the
  AD attribute \emph{mail} instead of the UCS attribute \emph{mail}.
\item The group membership cache has been improved (\ucsBug{28689},
  \ucsBug{28845})).
\item A LANMAN warning has been removed because the LM hash is no
  longer required (\ucsBug{28709}).
\end{itemize}



\section{Other changes}
\begin{itemize}
\item The ``known bugs'' message has been removed from
  \ucsName{univention-repository-addpackage},
  \ucsName{univention-repository-delpackage},
  \ucsName{univention-repository-merge},
  \ucsName{univention-directory-manager}, \ucsName{univention-join},
  \ucsName{univention-server-join},
  \ucsName{univention-run-join-scripts},
  \ucsName{univention-system-info}, and
  \ucsName{univention-pkgdb-scan} (\ucsBug{25379}).

\item An incompatibility between \ucsName{plymouth} and
\ucsName{sysvinit} which led to a crash of the boot splash has been
resolved (\ucsBug{28287}).

\item A quoting bug was fixed in the \ucsCommand{jitter} command
  (\ucsBug{16428}).

\item The \ucsName{ucs-test} suite is now released in the unmaintained archive
  section of UCS 3.1 (\ucsBug{28215}). Several tests have been adjusted to the current package versions
(\ucsBug{28356}, \ucsBug{28357}, \ucsBug{22937}, \ucsBug{21765},
\ucsBug{28360}, \ucsBug{28341}, \ucsBug{28342}, \ucsBug{28343},
\ucsBug{28348}, \ucsBug{28262}, \ucsBug{25818}, \ucsBug{27650},
\ucsBug{11431}).

\item A bug in \ucsName{univention-demo-configuration} was fixed,
  which resulted in duplicated display of the \ucsMenuEntry{Basic
    settings} dialogue (\ucsBug{27978}).

\item The following packages have been rebuild with python 2.6 support
only (\ucsBug{26852}):
\begin{itemize}
\item univention-nagios
\item univention-pkgdb
\item univention-package-template-python
\item univention-licence
\item univention-base-files
\item univention-config-registry
\item univention-python
\item univention-management-console-module-mrtg
\item univention-management-console-module-ipchange
\item univention-management-console-module-packages
\item univention-management-console-module-services
\item univention-management-console-module-vnc
\item univention-management-console-module-ucr
\item univention-management-console-module-reboot
\item univention-management-console-module-top
\end{itemize}

\item Configuration for password-protected access to the Bacula database
with the \emph{bacula} user has been added in the package
\ucsName{univention-bacula} (\ucsBug{16257}).

\item The boot loader GRUB has been updated to version 1.99 (\ucsBug{29445}).

\item The bootsplash has been disabled for the grub recovery mode boot
entry (\ucsBug{28287}).

\item A superfluous dependency on \ucsName{diff} has been removed in
  the \ucsName{univention-grub} package (\ucsBug{27010}).

\item The default ramdisk size for the net installer has been increased
(\ucsBug{28477}).

\item The permissions of certain logfiles and logfile directories have been restricted
(\ucsBug{11431}):
\begin{itemize}
\item /var/log/lastlog
\item /var/log/faillog
\item /var/log/dpkg.log
\item /var/log/bootstrap.log
\item /var/log/mail*
\item /var/log/samba
\item /var/log/univention
\end{itemize}

\item The automatic start of the NFS server is configurable through the
  \ucsUCRV{nfs/autostart} (\ucsBug{11559}).

\item The new \ucsUCRV{homedir/mount/required} defines whether a
successful mount of the users home directory is required for the logon
(\ucsBug{23637}).

\item The quota rpc service is now started, if a entry in
  \ucsFile{/etc/exports} exists (\ucsBug{25430}).

\item The packages \ucsName{apt-transport-https},
\ucsName{dpt-i2o-raidutils}, \ucsName{raidutils}, \ucsName{varmon},
\ucsName{cpqarrayd}, \ucsName{mpt-status}, \ucsName{php-gettext},
\ucsName{multipath-tools} and \ucsName{util-linux-locales} have been
added to the maintained repository (\ucsBug{28153}, \ucsBug{17073},
\ucsBug{27363}, \ucsBug{28484}, \ucsBug{8708}).

\item The Horde webmailer has been updated to version 4.0.8
\ucsBug{25792}, \ucsBug{29200}).

\item \ucsName{univention-debug} no longer logs error messages to Syslog
(\ucsBug{25469}).

\item \ucsName{univention-system-info} has been modified to backup
existing archives before creating the sysinfo archive (\ucsBug{21497}).

\item \ucsName{univention-skel} has been modified to provide a \ucsFile{.profile},
which executes the personal \ucsFile{.bashrc} (\ucsBug{15754}).

\item Descriptions for the \ucsUCR{} variables have been added to
\ucsName{univention-log-collector} (\ucsBug{22537}).

\item Support for ldap/server/addition has been added to
\ucsName{univention\_policy\_result} and
\ucsName{univention-directory-policy} (\ucsBug{19428}).

\end{itemize}

\subsection{ucslint}
\begin{itemize}
\item A test for unquoted calls of \ucsCommand{eval \$(ucr shell)} has
  been added (\ucsBug{18443}).
\item The use of deprecated commands like
  \ucsCommand{univention-baseconfig} and \ucsCommand{univention-admin}
  now emits a warning (\ucsBug{23566}).
\item The use of globbing patterns instead of regular expressions in
  \ucsUCR{} files produces a warning
  (\ucsBug{13015}).
\item Fragile comparisons with None in Python code produce a warning
  (\ucsBug{22487}).
\item A warning is printed when Debian source files are not prefixed
  with the package name (\ucsBug{23482}).
\item In \ucsUCR{} template info files the fields \emph{Mode},
  \emph{User}, \emph{Group}, \emph{Preinst}, and \emph{Postinst} are
  now also checked (\ucsBug{14070}, \ucsBug{14201}).
\item Using UCRWARNING before a file type specific magic strings
  produces a warning (\ucsBug{21431}).
\item \ucsUCR{} templates not using any \ucsUCR{} variables produce a
  warning (\ucsBug{22434}).
\item Tests for files in sub-directories used by version control
  systems like GIT and Subversion and automatically generated files
  are skipped (\ucsBug{22470}, \ucsBug{22482}, \ucsBug{22498}).
\item Translations of multi line strings are now also checked for
  incorrectly substitued variables (\ucsBug{22479}).
\item Several coding and packaging problems in \ucsCommand{ucslint}
  have been fixed (\ucsBug{22482}, \ucsBug{22484}, \ucsBug{27429},
  \ucsBug{27444}).
\item The test for \ucsUCR{} \verb|is_true()| and \verb|is_flase()|
  has been improved (\ucsBug{22485}).
\item Comments are better detected and are no longer checked for
  issues (\ucsBug{23465}).
\item \ucsCommand{dh-umc-module-install} is now recognized as a
  program indirectly installing join and template files
  (\ucsBug{24816}).
\item The test for checking Python issues in hash-bang lines has been
  improved to better handle white spaces (\ucsBug{25766}).
\item Most error messages now consistently include the name of the
  file, where the error was found (\ucsBug{27889}).
\end{itemize}
